\documentclass[12pt,a4paper]{article}
\usepackage[utf8]{inputenc}
\usepackage{amsmath}
\usepackage{amsfonts}
\usepackage{amssymb}
\usepackage{amsthm}
\usepackage{tikz-cd}
\usepackage{graphicx}
\author{Marco Bertenghi}

\newtheorem{lem}{Lemma}[section]
\newtheorem{thm}{Theorem}[section]
\newtheorem{prop}{Proposition}[section]
\newtheorem{cor}{Corollary}[section]
\theoremstyle{definition}
\newtheorem{defn}{Definition}[section]
\newtheorem{exmp}{Example}[section]
\newtheorem{rem}{Remark}[section]

%http://tikzcd.yichuanshen.de/#eyJub2RlcyI6W3sicG9zaXRpb24iOlsxLDBdLCJ2YWx1ZSI6IlgifSx7InBvc2l0aW9uIjpbNCwwXSwidmFsdWUiOiJZIn0seyJwb3NpdGlvbiI6WzQsMl0sInZhbHVlIjoiWV57Kip9In0seyJwb3NpdGlvbiI6WzEsMl0sInZhbHVlIjoiWF57Kip9In0seyJwb3NpdGlvbiI6WzAsMF0sInZhbHVlIjoiWV4qIn0seyJwb3NpdGlvbiI6WzUsMl0sInZhbHVlIjoiWF4qIn0seyJwb3NpdGlvbiI6WzAsMl0sInZhbHVlIjoiWV57KioqfSJ9XSwiZWRnZXMiOlt7ImZyb20iOjAsInRvIjoxLCJ2YWx1ZSI6ImcifSx7ImZyb20iOjEsInRvIjoyLCJsYWJlbFBvc2l0aW9uIjoibGVmdCIsInZhbHVlIjoiSl9ZIiwidGFpbCI6Imhvb2thbHQifSx7ImZyb20iOjMsInRvIjoyLCJsYWJlbFBvc2l0aW9uIjoibGVmdCIsInZhbHVlIjoiZ157Kip9In0seyJmcm9tIjowLCJ0byI6MywidmFsdWUiOiJKX1giLCJsYWJlbFBvc2l0aW9uIjoicmlnaHQiLCJ0YWlsIjoiaG9va2FsdCIsImhlYWQiOiJ0d29oZWFkcyJ9LHsiZnJvbSI6MCwidG8iOjQsInZhbHVlIjoiZiIsImxhYmVsUG9zaXRpb24iOiJyaWdodCIsInRhaWwiOiJob29rIn0seyJmcm9tIjoyLCJ0byI6NSwibGFiZWxQb3NpdGlvbiI6ImxlZnQiLCJ2YWx1ZSI6ImZeKiIsInRhaWwiOiJob29rYWx0In0seyJmcm9tIjozLCJ0byI6NiwiaGVhZCI6bnVsbCwibGFiZWxQb3NpdGlvbiI6InJpZ2h0IiwidmFsdWUiOiJmXnsqKn0ifSx7ImZyb20iOjQsInRvIjo2LCJ0YWlsIjoiaG9va2FsdCIsInZhbHVlIjoiSl97WV4qfSIsImxhYmVsUG9zaXRpb24iOiJyaWdodCJ9XX0=

\begin{document}
\section{Generalization of Hahn-Banach}
Let $X$ be a vector space over $\mathbb{R}$ and $Y \subset X$ a  linear subspace. Let $p: X \to \mathbb{R}$ be a sublinear functional and $f: Y \to \mathbb{R}$ linear with $f \leq p$ on $Y$. \\ \\
Consider now $G \subset \mathcal{L}(X)=\mathcal{L}(X,X)$ a subset of bounded linear operators with the properties that id$_X \in G$ and for all $A,B \in G, AB \in G$ and moreover $AB=BA$. Assume that for all $A \in G$ we have $p(Ax) \leq p(x)$ for all $x \in X$, $Ay \in Y$ and $f(Ay)=f(y)$ for all $y \in Y$. \\
\\
\textbf{Claim:} There exists $F: X \to \mathbb{R}$ linear with $F_{ \mid Y} = f, F \leq p$ on $X$ and $F(Ax)=F(x)$ for all $x \in X$ and $A \in G$. 
\begin{proof}
Given the hint to consider $q(x):= \inf_{A_1, \dots , A_n} \frac{1}{n} p(A_1x + \dots + A_nx)$ for $x \in X$. Here the infimum is taken over finitely many $A_i \in G$. Then it is clear that $q$ is sublinear because all the $A_i:X \to X$ are linear and $p$ itself is sublinear. \\
\\
Moreover for $y \in Y$ and $A_1, \dots ,  A_n \in G$ where $n \in \mathbb{N}$ we have
\begin{align*}
f(A_1y + \dots + A_ny)= f(A_1y) + \dots + f(A_ny)= n f(y) 
\end{align*}
But also $f(A_1y + \dots + A_ny) \leq p(A_1y + \dots + A_ny)$ by assumption and thus we have $f(y) \leq \frac{1}{n}p(A_1y + \dots + A_ny)$ for all $A_1, \dots ,  A_n \in G$. Taking the infimum we conclude that $f(y) \leq q(y)$ on $Y$. 
\\\\
By \textbf{Hahn-Banach Theorem} there exists $F: X \to \mathbb{R}$ linear with $F_{ \mid Y} = f$ and $F(x) \leq q(x)$ for all $x \in X$. However since we also have that $p(A_1x + \dots + A_nx) \leq p(A_1x) + \dots + p(A_nx) \leq n p(x)$ it also follows that $q \leq p$ on $X$ which takes care of the first part of the claim. \\
\\
Let now $A_1, \dots , A_n \in G$ be arbitrary, then we have that \begin{align*}
q(Ax-x)&\leq \frac{1}{n}p (A_1(Ax-x) + \dots + A_n(Ax-x)) \\
&= \frac{1}{n} p(A_1Ax + \dots + A_nAx - (A_1x + \dots + A_nx)) \\
& = \frac{1}{n}p (A A_1x + \dots + AA_nx - (A_1x + \dots + A_nx))
\end{align*}
If we now apply this to the special case where $A_i = A^{i-1} \in G$ for $i= 1 , \dots ,  n$ where $A^0=id_X \in G$ we obtain from the emerging telescoping sum that for all $n \in \mathbb{N}$ we have 
\begin{align*}
q(Ax-x) \leq \frac{1}{n}p(A^nx -x) \leq \frac{1}{n}( p(A^nx)+p(-x)) \leq \frac{1}{n}(p(x)+ p(-x))  
\end{align*}
Passing this statement to the limit as $n \to \infty$ we obtain that $q(Ax-x) \leq 0$. This implies that $F(Ax-x) \leq 0$ because $F \leq q$ on $X$. Thus we've got the inequality
\begin{align*}
F(Ax) \leq F(x)
\end{align*}
But since both $F$ and $A \in G$ are linear we obtain that 
\begin{align*}
-F(Ax)=F(-Ax)=F(A(-x)) \leq F(-x) = -F(x)
\end{align*}
Multiplying by $(-1)$ we get that $F(Ax) \geq F(x)$ and thus $F(Ax)=F(x).$
\end{proof}
\newpage
\section{Reflexivity}
Let $X$ and $Y$ be Banach spaces with an isometric linear map $f: X \to Y^*$ such that $f^*: Y^{**} \to X^*$ is also isometric. Moreover, let $X$ be reflexive. Show that there exists isometric isomorphisms $Y \cong X^*$ and $X \cong Y^*$.
\begin{proof} First consider the diagram below, seriously, it's pretty sweet. 
\begin{center}
\begin{tikzcd}
Y^* \arrow[dd, "J_{Y^*}"', hook'] & X \arrow[rrr, "g"] \arrow[dd, "J_X"', two heads, hook'] \arrow[l, "f"', hook] &  &  & Y \arrow[dd, "J_Y", hook'] &  \\
 &  &  &  &  &  \\
Y^{***} & X^{**} \arrow[rrr, "g^{**}"] \arrow[l, "f^{**}"'] &  &  & Y^{**} \arrow[r, "f^*", hook'] & X^*
\end{tikzcd}

\end{center}
In the previous exercise sheet I have already shown that the above diagram commutes, i.e. we have $g^{**} \circ J_X = J_Y \circ g$ under the assumption that $g$ is a linear map. \textit{(A similar proof can be found in the proof of Thm 4.3.3.)} \\
\\
Now we recall that $J_X, J_Y$ are linear isometries and quite generally, the composition of (linear) isometries is again an  (linear) isometry. Moreover, isometries are always continuous and injective. Finally, if we have a bijective isometry, then quite trivially the inverse of said isometry is again an isometry. 
\\
\\
Recall from \textbf{Thereom 4.3.4.} That for a Banach Space $X$ we have $X$ is reflexive if and only if $X^*$ is reflexive. We will need that later. 
\\\\
Also recall from \textbf{Exercise Sheet 6 Exercise 5} that if $X,Y$ are Banach spaces and $T: X \to Y$ is a continuous isomorphism then $X$ is reflexive if and only if $Y$ is reflexive. 
\\\\
\textbf{Claim 1:} Let $f: X\to Y$ be an linear, continuous and injective map, then $f^*: Y^* \to X^*$ is surjective. 
\\\\
\textbf{Proof of Claim 1:} Let $x^* \in X^*$ be arbitrary. Since $f$ is linear Im$(f)\subset Y$ is a linear subspace. We define
\begin{align*}
\Psi: \begin{cases} \text{Im}(f) & \longrightarrow \mathbb{K} \\
f(x) & \longmapsto \Psi(f(x))=x^*(x) \end{cases} 
\end{align*} 
We notice that $\Psi$ is well defined because $f$ is assumed to be injective, in particular $f$ has a 1 to 1 correspondence to its image. Clearly $\Psi$ is linear, because both $f$ and $x^*$ are linear. Further we have \begin{align*}
\| \Psi(f(x)) \|  \leq \|x^*\| \| x \| \text{ for all } x \in X
\end{align*}
Hence we have that $\Psi$ is continuous and thus $\Psi \in \text{Im}(f)^*$. By \textbf{Corollary 4.2.7. to Hahn Banach} we have that $\Psi$ extends continuously and linearly on $Y$ i.e. $\Psi \in Y^*$. But then by definition of the adjoint we have that \begin{align*}
f^*(\Psi)= \Psi \circ f = x^*
\end{align*}
which entails that $f^*$ is surjective as claimed. \hfill $\Box$ \\\\
Applying this result now to our $f:X \to Y^{*}$ which is a linear isometry we obtain that $f^*:Y^{**} \to X^*$ is surjective. Moreover, by assumption $f^*$ is an linear isometry we now have that $f^*$ is in fact an continuous isomorphism between $Y^{**}$ and $X^*$ and since $X^*$ is reflexive, so is $Y^{**}$.
\\\\
Since by assumption $f^*$ is an linear isometry (in particular injective) we also get that $f^{**}: X^{**} \to Y^{***}$ is surjective. \\ But now we have that $J_{Y^*} \circ f = f^{**} \circ J_X: X \to Y^{***}$ is surjective as the composition of surjective maps \textit{(recall that $X$ is reflexive by assumption)}, thus $J_{Y^*}$ has to be surjective too. This entails that $Y^*$ is reflexive and this is the case if and only if $Y$ is reflexive, i.e. $J_Y$ is an isometric isomorphisms. \\
\\
This now shows that $f^* \circ J_Y : Y \to X^*$ is an isometric isomorphism. 
\\\\
In order to show that $X \cong Y^*$ we start with another claim. 
\\\\
\textbf{Claim 2:} Let $f:X \to Y$ be an isometric isomorphism. Then $f^*: Y^* \to X^*$ is a linear isometry. In Particular by the first \textbf{Claim 1} it follows that $f^*$ is an isometric isomorphism. 
\\\\
\textbf{Proof of Claim 2:} We have by definition 
\begin{align*}
\|f^*(y^*)\|_{X^*} = \| y^* \circ f \|_{X^*} = \sup_{\substack{ x \in X \\ \|x \| \leq 1 }} | y^*(f(x))|
\end{align*}
Since $f: X \to Y$ is an isomorphism we can find for all $y \in Y$ with $\|y \| \leq 1$ an $x \in X$ such that $f(x)=y$, but then we have because $f$ is also isometric that \begin{align*}
\|y\|_Y = \|f(x)\|_Y = \|x\|_X \leq 1 
\end{align*}
Similarly if $x \in X$ such that $\|x\| \leq 1$ then also $\|f(x)=y\| = \|x \| \leq 1$. Thus we obtain that 
\begin{align*}
\|f^*(y^*)\|_{X^*} = \sup_{\substack{ x \in X \\ \|x \| \leq 1 }} | y^*(f(x))| = \sup_{ \substack{ y \in Y \\ \| y \| \leq 1}} y^*(y)| = \| y^*\|_{Y^*}
\end{align*}
That is $f^*$ is an isometry \hfill $\Box$ \\
\\
Now if we bring this all together, we have that $J_X: X \to X^{**}$ and $J_{Y^*}: Y^* \to Y^{**}$ are isometric isomorphisms, moreover, by our efforts above we have that $f^{**}: X^{**} \to Y^{***}$ is an isometric isomorphism.
 \\
\\
\textbf{Remark:} Alternatively we could also have said that because $f^{**} \circ J_X = J_{Y^*} \circ f$,  it follows immediately that $f^{**}$ is an isometry, because $f^{**} = J_{Y^*} \circ f \circ (J_X)^{-1}$ is the composition of linear isometries. 
\\\\
We conclude that $(J_{Y^*})^{-1} \circ f^{**} \circ J_X : X \to Y^{*}$ is an isometric isomorphism, that is $X \cong Y^*$ which concludes the proof. 
\end{proof}
\newpage
\section{Hellinger-Toeplitz Theorem}
Let $H$ be a Hilber space and $A: H \to H$ linear and symmetric, i.e. \begin{align*}
\langle y, Ax \rangle = \langle Ay,x \rangle \ \text{ for all } x,y \in H.
\end{align*}
Show that then $A$ is bounded. 
\begin{proof}
We first recall \textbf{Theorem 5.1.4.}
\\\\
\textbf{Theorem 5.1.4. (Banach-Steinhaus):} Let $X$ be a Banach space, $Y$ a normed space and $\mathcal{F} \subset \mathcal{L}(X,Y)$. Assume that for all $x \in X$ there exists $c_x \geq 0$ such that 
\begin{align*}
\sup_{T \in \mathcal{F}} \| Tx \| \leq c_x 
\end{align*}
Then there exists $c \geq 0$ with 
\begin{align*}
\sup_{T \in \mathcal{F}} \| T \| \leq c 
\end{align*}
We take $X=H$ (recall that every Hilbert Space is also a Banach Space) and $Y= \mathbb{K}$. For $y \in H$ we define 
\begin{align*}
f_y :H \to \mathbb{K}, \ f_y(x):= \langle Ay, x \rangle
\end{align*}
Then we have that $f_y$ is linear because by our definition of the scalar product we have that $\langle \cdot ,\cdot  \rangle$ is linear in its second argument. Moreover the function $f_y$ is continuous, because it's defined as an inner product. Thus we have that $f_y \in \mathcal{L}(H, \mathbb{K})$.\\
\\
Let us now define
\begin{align*}
\mathcal{F}:= \lbrace f_y \in \mathcal{L}(X, \mathbb{K}) \mid \| y \| =1 \rbrace \subset \mathcal{L}(X, \mathbb{K})
\end{align*}
Let now $f_y \in \mathcal{F}$ be arbitrary. Then for $x \in H$ we have by the Cauchy-Schwarz inequality
\begin{align*}
\| f_y(x)\| = | \langle Ay, x \rangle | = | \langle y, Ax \rangle |  \leq \| y \| \| Ax \| = \|Ax \| =:c_x \\
\implies \sup_{ f_y \in \mathcal{F}} \| f_y(x) \| \leq c_x 
\end{align*}
By \textbf{Banach-Steinhaus} there exists $c \geq 0$ such that $\sup_{f_y \in \mathcal{F}} \| f_y \| \leq c$ 
\newpage
We now have established the existence of a constant $c \geq 0$ such that 
\begin{align*}
\sup_{f_y \in \mathcal{F}} \sup_{ \|x \| =1 } \| f_y(x)\| \leq c
\end{align*}
Now we obtain that 
\begin{align*}
\| Ax \|^2 = \langle Ax, Ax \rangle = \|x \| \langle A(x/ \|x \|), Ax \rangle = \|x \| f_{ x/\|x\|} (Ax)
\end{align*}
We have for $y= x/\|x\|$ that $\|y\|=1$ and we use the bound that we established through the Banach-Steinhaus theorem, i.e.
\begin{align*}
f_{x / \|x \|} (Ax) \leq  c  \|Ax \|, \text{ where } c= \sup_{f_y \in \mathcal{F}} \| f_y \| 
\end{align*}
Thus we have
\begin{align*}
\| Ax \|^2 \leq c \|x \| \| Ax \| \implies \|Ax\| \leq c \|x \|
\end{align*}
That is $A$ is bounded. 
\end{proof}
\newpage
\section{Exercise 5: Bilinear functionals}
Let $X$ be a normed vector space over $\mathbb{K}$. A bilinear functional on $X$ is a map $B: X \times X \to \mathbb{K}$ such that for all $x,y \in X$ the maps $B(x, \cdot): X \to \mathbb{K}$ and $B( \cdot , y) :X \to \mathbb{K}$ are linear functionals on $X$.
\\
\\
a) Let $X$ be a Banach space and $B$ a bilinear functional on $X$ which is continuous in each variable separately, i.e. for every fixed $x,y \in X$, the maps $B(x, \cdot)$ and $B( \cdot ,y)$ are continuous. Show that there exists a constant $C>0$ such that $|B(x,y) | \leq C \| x \| \| y \|$ for all $x,y \in X$. Conclude that $B$ is continuous with respect to the norm $\|(x,y)\|:= \|x \| + \|y \|$ on $X \times X$.
\begin{proof}
We define 
\begin{align*}
\mathcal{F}:= \lbrace B( \cdot , y) : \|y \| = 1 \rbrace 
\end{align*}
By definition $B( \cdot ,  y)$ is linear because it's a bilinear functional. Further by assumption it is continuous in it's first argument, thus we have for all $x \in X$ that
\begin{align*}
|B(x,y)| \leq c_y \|x\| 
\end{align*}
Hence we indeed have that $\mathcal{F} \subset \mathcal{L}(X, \mathbb{K})$. Further, because $B$ is also linear in its second argument we also have for any $x \in X$ and any $y \in X$ with $\|y\|=1$ that 
\begin{align*}
|B(x,y)| \leq c_x \|y \| = c_x 
\end{align*}
Thus we conclude that
\begin{align*}
\sup_{T \in \mathcal{F}} | T(x)| \leq c_x
\end{align*}
Thanks to Banach-Steinhaus we now know that there exists a constant $c \geq 0$ such that
\begin{align*}
\sup_{T \in \mathcal{F}} \|T \| \leq c 
\end{align*}
Let now $x,y \in X$ be arbitrary. 
\begin{align*}
|B(x,y)| &= \| x \| \|y \| |B(x/\|x \|, y / \|y \|) |\leq \| x \| \|y \| \sup_{ \| \xi \| = 1} |B( \xi , y / \|y \|)| \\
&\leq \sup_{ \| \zeta \| =1} \sup_{ \| \xi \| = 1} | B(\xi, \zeta)| = \|x \| \| y \| \sup_{ \| \zeta \| = 1} | B( \cdot , \zeta)|  \leq c \| x \| \| y \| 
\end{align*}
Which takes care of the first part of the claim. 
\newpage
Next we need to show that with respect to the norm $\| (x,y) \| := \|x \| + \|y\| $ on $X \times X$ the function $B$ is continuous. 
\\\\
To this extent consider the sequence $(x_n, y_n)_{n \in \mathbb{N}} $ in $X \times X$ and assume that this sequence converges to $(x,y) \in X \times X$. Then by definition of the norm on $X \times X$ we have that
\begin{align*}
0\overset{n \to \infty}\longleftarrow\|(x_n,y_n)-(x,y)\| = \| (x_n-x,y_n-y)\| = \|x_n-x \| + \|y_n -y\| 
\end{align*}
Which shows that $x_n \to x$ and $y_n \to Y$ in $X$. In particular the sequence $y_n$ is bounded as a convergent sequence in $X$, i.e. there exists $M \geq 0$ such that $\| y_n \| \leq M$  for all $n \in \mathbb{N}$. Finally we obtain that
\begin{align*}
|B(x_n,y_n)-B(x,y)| &= |B(x_n,y_n)-B(x,y_n) + B(x,y_n) - B(x,y)| \\
&= |B(x_n-x, y_n) + B(x,y_n-y)| \\
& \leq |B(x_n-x, y_n)| + |B(x, y_n-y)| \\
& \leq c \| x_n-x\| \|y_n\| + c \|x\| \|y_n-y \| \\
& \leq cM \|x_n -x \| + \|x \| \|y_n-y \| \to 0 \text{ as } n \to \infty 
\end{align*}
This shows that that $B$ is continuous. 
\end{proof}
b) Let $\mathcal{P}$ be the vector space of real polynomials in one variable,  equipped with the norm $\|p\| = \int_0^1 |p(t)| dt$ for $p \in \mathcal{P}$. Let
\begin{align*}
B(p,q)= \int_0^1 p(t)q(t)dt
\end{align*}
Show that $B$ is a (real valued) bilinear functional on $\mathcal{P}$ which is continuous variable separately, but that $B$ is not continuous on $\mathcal{P} \times \mathcal{P}$. 

\end{document}