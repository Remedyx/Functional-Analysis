
\documentclass[12pt,a4paper]{article}
\usepackage[utf8]{inputenc}
\usepackage{amsmath}
\usepackage{amsfonts}
\usepackage{amssymb}
\usepackage{amsthm}
\usepackage{tikz-cd}
\usepackage{graphicx}
\author{Marco Bertenghi}

\newtheorem{lem}{Lemma}[section]
\newtheorem{thm}{Theorem}[section]
\newtheorem{prop}{Proposition}[section]
\newtheorem{cor}{Corollary}[section]
\theoremstyle{definition}
\newtheorem{defn}{Definition}[section]
\newtheorem{exmp}{Example}[section]
\newtheorem{rem}{Remark}[section]

\newcommand{\wto}{\rightharpoonup}
\newcommand{\wstarto}{\overset{*}\rightharpoonup}

\begin{document}
\section{Quotient spaces}
Let $X$ be a Banach space and $M \subset X$ a closer linear subspace. Let $\sim$ be the equivalence relation defined through $x \sim y$ if and only if $x-y \in M$ and let $X/M$ be the quotient space with respect to $\sim.$
\\\\
\textbf{a)} For $[x] \in X/M$ let 
\begin{align*}
\|[x]\|_{X/M}:= \inf_{y \in [x]} \| y \|_X = \inf_{y \in M} \|x-y\|_X
\end{align*}
Show that $\| \cdot \|_{X/M}$ is a norm on $X/M$. 
\begin{proof}
Indeed we have that $\| \cdot \|_{X/M}$ is well-defined, it is in particular finite because $\| \cdot \|_X$ is a norm on $X$. Moreover we have
\begin{align*}
\| [x]\|_{X/M} = \inf_{y \in M} \|x-y\| = 0
\end{align*}
Thus we can construct a sequence $y_n$ in  $M$ such that $\|x-y_n\| \to 0 $ as $n \to \infty$. But since $M \subset X$ is closed we have that $y_n \to x \in M$, henceforth $[x]=0$. 
\\\\
Next we have
\begin{align*}
\| [x] + [y]\|_{X/M} &= \| [x+y]\|_{X/M} = \inf_{v \in M} \| x+y-v\| \\
&= \inf_{v \in M} \| x-v + y-v + v\| \\ 
& \leq \inf_{v \in M} \| x-v\| + \inf_{v \in M} \|y-v\| + \inf_{v \in M} \|v\| \\
& = \inf_{v \in M} \|x-v\| + \inf_{v \in M} \|y-v\|   \\
& 
= \| [x]\|_{X/M} + \|[y]\|_{X/M} 
\end{align*}
Finally we have for all $\lambda \in \mathbb{K}$
\begin{align*}
\| \lambda [x]\|_{X/M} &=  \|[ \lambda x] \|_{X/M} = \inf_{y \in M} \| \lambda x -y\|  \\
& = \inf_{ \lambda y  \in M} \| \lambda x - \lambda y\| \\
&= \inf_{u \in M} \| \lambda x - \lambda u\| \tag{*} \\
 & = \inf_{u \in M} | \lambda | \|x-u\| = | \lambda | \| [x] \|_{X/M} 
\end{align*}
 Where we used in (*) that $\varphi_\lambda : M \to M$ given by $ \varphi_\lambda(x)= \lambda x$ is a bijection for all $\lambda \in \mathbb{K}$. Thus we have shown that $\| \cdot \|_{X/M}$ is a norm on $X/M$. 
\end{proof}
\newpage
\noindent \textbf{b)} Show that the projection $\pi : X \to X/M, \ x \mapsto [x]$ is continuous. 
\begin{proof}
Evidently, the function $\pi$ is linear because we have by definition that
\begin{align*}
[x+y]&=[x]+[y] \\
\lambda [x] &= [\lambda x]  
\end{align*}
For all $x,y \in X, \lambda \in \mathbb{K}$. Moreover we have 
\begin{align*}
\| \pi(x) \|_{X/M} = \| [x] \|_{X/M} = \inf_{y \in M} \| x-y\| \leq \inf_{y \in M} \| x\| + \underbrace{\inf_{y \in M} \| y \|}_{=0} = \|x\|_X 
\end{align*}
That is $\pi$ is a bounded linear operator and thus continuous. 
\end{proof}
\noindent \textbf{c)} Show that $X/M$ is complete.
\begin{proof}
Recall from Exercise Sheet 1 Exercise 4: Normed spaces and Banach spaces where we have shown
\\\\
\textbf{Theorem:} Let $(X, \| \cdot \|)$ be a normed space. $(X, \| \cdot \|)$ is a Banach space if and only if, all absolutely convergent series are also convergent, i.e. if 
\begin{align*}
\sum_{n=0}^\infty \|x_n\| < \infty \implies \sum_{n=0}^\infty x_n < \infty 
\end{align*}
Thanks to the above theorem in order to prove that $X/M$ is a Banach space it is enough to show that every series in $X/M$ that converges absolutely also converges in $X/M$. 
\\\\
To this extent let us take an arbitrary sequence $([x_n])_{n \in \mathbb{N}}$ in $X/M$ such that its series is absolutely convergent, i.e. we have 
\begin{align*}
\sum_{n=0}^\infty \|[x_n]\|_{X/M} < \infty 
\end{align*}
By definition of the quotient norm (via the inf) we have for all $n \in \mathbb{N}$ the existence of some $y_n \in M$ such that $\|x_n-y_n \| \leq \|[x_n]\|_{X/M} + 1/2^n$. Since we've already established the convergence of the right hand side, the convergence of the left hand side is immediate by domination. But $X$ is a Banach space, thus it follows from the Theorem above that
\begin{align*}
\sum_{n=0}^\infty (x_n-y_n)=x < \infty 
\end{align*}
We now claim that the series over the sequence $([x_n])_{n \in \mathbb{N}}$ in $X/M$ converges to $[x] \in X/M$. 
\newpage
Indeed, let us consider for $N \in \mathbb{N}$ 
\begin{align*}
\left\| [x]- \sum_{n=0}^N [x_n] \right\|_{X/M} = \left\| [x - \sum_{n=0}^N x_n] \right\|_{X/M}  \leq \left\| x- \sum_{n=0}^N x_n \right\|_X \to 0 
\end{align*}
as $N \to \infty$. Where the last inequality just follows by the very definition of the quotient norm. 
\end{proof} 
\noindent \textbf{d)} Let $X$ and $Y$ be Banach spaces and $T \in \mathcal{L}(X,Y)$. \\ We define the kernel of $T, \ker T := \lbrace x \in X : Tx = 0 \rbrace$ and the range of $T$, ran$(T):= \lbrace Tx: x \in X \rbrace \subset Y$. Let $\iota : \text{ran}T \to Y , x \mapsto x$ be the inclusion map. Show that then there exists a bijective operator $\hat{T} \in \mathcal{L}(X/ \ker T, \text{ran}T)$ with $T = \iota \hat{T} \pi$ and $\| \hat{T} \| = \|T \|$. 
\begin{proof}
We have indeed that $\ker T \subset X$ is a closed linear subspace. Now we define 
\begin{align*}
\hat{T}: X/\ker T \to \text{ran}T, \  [x] \mapsto T(x)
\end{align*}
We notice that that $\hat{T}$ is well-defined. Indeed, let $[x], [y] \in X/ \ker T$ be such that $[x]=[y]$, this is the case if and only if $x \sim y$ i.e. $x-y \in \ker T$, or equivalently $x \in y + \ker T$. This entails that we can write $x= y + t$ for some $t \in \ker T$. Thus we get
\begin{align*}
\hat{T}([x])=T(x)= T(y+t)=T(y)+\underbrace{T(t)}_{=0} = T(y)=T([y])
\end{align*}
Which shows that $\hat{T}$ is well-defined. Moreover, $\hat{T}$ is by definition (defined through $T$) surjective onto the range of $T$. Furthermore $\hat{T}$ is injective, let $[x],[y] \in X/\ker T$ be such that 
\begin{align*}
\hat{T}([x])=\hat{T}([y]) \iff T(x)=T(y)& \implies T(x-y)=0 \\
 &\iff x-y \in \ker T \\
 & \iff x \sim y \\
 & \iff [x]=[y]
\end{align*} 
The linearity of $\hat{T}$ follows immediatly by the linearty of $\pi$ and the linearty of $T$. Finally, we have that $\pi$ is continuous, and by assumption $T$ is continuous too. 

\newpage
We have seen that $\|[x]\|_{X/\ker T} \leq \|x\|_X$ and by the continuity of $T$ we obtain for some $C>0$ that
\begin{align*}
\| \hat{T}([x])\|_Y = \| T(x)\|_Y \leq C \|x\|_X
\end{align*}
let us now set $\tilde{C}:= \max (1, C)$. This yields that
\begin{align*}
\| \hat{T}([x])\|_Y \leq C\|x\|_X \leq \tilde{C}\|x\|_X = \tilde{C} \|x-0\|_X
\end{align*}
Since $0 \in \ker T$ we conclude that
\begin{align*}
\| \hat{T}([x]) \|_Y \leq \inf_{t \in \ker T} \| x-t\| = \|x\|_{X/\ker T}
\end{align*}
showing that indeed $\hat{T} \in \mathcal{L}(X/ \ker T, \text{ran}T)$. \\
\\
Further, by the very definition of $ \hat{T}$ we have that $T= \iota \hat{T} \pi$. 
\\\\
Lastly, we observe that since $0 \in \ker T$ we have that if $\|x\|_X \leq 1$ for an arbitrary $x \in X$, then $\|x-0\|_X=\|x\|_X \leq 1$ and thus by definition of the infimum as the greatest lower bound we must also have $\inf_{y \in \ker T} \|x-y\| \leq 1$.
\\\\
Conversely, if $\inf_{y \in \ker T} \|x-y \| \leq 1$. Then again by definition of the greatest upper bound we must also have that $\|x-0\|=\|x\| \leq 1$. This shows that
\begin{align*}
\| \hat{T}\| &= \sup_{ \substack{ [x] \in X/ \ker T \\ \| [x]\|_{X / \ker T} \leq 1}} \| \hat{T}([x])\|_Y = \sup_{ \substack{ [x] \in X/ \ker T \\ \| [x]\|_{X / \ker T} \leq 1}} \| T(x)\|_Y \\
& = \sup_{\substack{ x \in X \\ \|x\|_X \leq 1 }} \| T(x)\|_Y = \|T\|
\end{align*}
\end{proof}
\newpage
\section{Reflexivity and weak convergence}
\textbf{a)} Let $X$ be a normed space and $X^*$ its dual. Let $(L_n)_{n \in \mathbb{N}}$ be a sequence in $X^*$ and $(x_n)_{n \in \mathbb{N}}$ a sequence in $X$. Let $L_n \overset{*}\wto L$ in $X^*$ with respect to the weak-$*$ topology and $x_n \to x$ with respect to the norm in $X$. Show that if $X$ is reflexive, then $L_n(x_n) \to L(x)$ as $n \to \infty$. 
\begin{proof}
We recall that the space $X^{**}$ is always a Banach space, since by assumption $X$ is reflexive we have $X \cong X^{**}$, i.e. $X$ is a Banach space as well. Thus we know that the reflexivity of $X$ is equivalent to saying that $X^*$ is reflexive.
\\\\
\textbf{Claim:} Let $X$ be a reflexive Banach space, then weak-$*$ convergence of a sequence $(f_n)_{n \in \mathbb{N}}$ in $X^*$ implies weak convergence. 
\\\\
\textbf{Proof of Claim:} Let $(f_n)_{n \in \mathbb{N}}$ be a sequence in $X^*$ that converges towards $f \in X^*$ w.r.t. $\tau_W^*$, this is the case if and only if $f_n (x) \to f(x)$ for all $x \in X$ as $n \to \infty$. 
\\\\
We want to show that $f_n \wto f$ in $X^*$, that is by \textbf{Lemma 6.2.1.} equivalent to showing that $\lambda (f_n) \to \lambda (f)$ as $n \to \infty$ for all $\lambda \in X^{**}$.
\\\\
Since $X$ is reflexive, the canonical inclusion $J_X: X \to X^{**}$ is surjective, that is to say that for every $\lambda \in X^{**}$ we find an $x \in X$ such that $J_X(x)= \lambda$. But then by the definition of the canonical inclusion we have
\begin{align*}
\lambda (f_n)= J_X(x)(f_n) = f_n(x) \to f(x) = J_X(x)(f)= \lambda (f) 
\end{align*}
which gives the weak convergence. $\hfill \Box$
\\\\
Next we recall that weakly convergent sequences are always bounded. By our efforts above we now have for the sequence $(L_n)_{n \in \mathbb{N}}$ in $X^*$ that $L_n \wstarto L$ in $X^*$ implies that $L_n \wto L$ in $X^*$ and the sequence $(L_n)_{n \in \mathbb{N}}$ is bounded. Let $M >0$ be a constant such that $\|L_n\|_{X^*} \leq M$ for all $n \in \mathbb{N}$, then 
\begin{align*}
|L_n(x_n)-L(x)| &= |L_n(x_n)-L_n(x) + L_n(x)-L(x)| 
\\
& \leq | L_n (x_n-x)| + |L_n(x)-L(x)|  \\
& \leq \|L_n\|_{X^*} \|x_n-x\|_X + |L_n(x)-L(x)| \\
& \leq M \|x_n-x\|_X + |L_n(x) - L(x)|
\end{align*}
Letting $n \to \infty$ the right hand side converges to zero since $x_n \to x$ with respect to the norm in $X$ and $L_n$ converges in the sense of weak-$*$ converges towards $L$ \textit{(see characterization at begin of proof of claim)}. 
\end{proof}
\newpage
\noindent \textbf{b)} Let $X$ and $Y$ be normed vector spaces and let $T: X \to Y$ be linear. Furthermore let $T$ be such that \begin{align*}
x_n \wto 0 \text{ in } X \implies Tx_n \wto 0 \text{ in } Y
\end{align*}
Let $X$ be reflexive, show that then $T$ is bounded. 
\begin{proof}
For the sake of contradiction, assume that $T$ is not bounded, this means that for every $n \in \mathbb{N}$ there exists $x_n \in X$ such that  $\|Tx_n \| \geq n^3 \| x_n\|$. We can easily choose $x_n$ to be normed to $1$. Next we define $\tilde{x_n} := x_n/n$ and notice that since $x_n$ is normed that $\| \tilde{x_n} \|_X \to 0$ as $n \to \infty$. 
\\\\
Moreover we have
\begin{align*}
\| T \tilde{x_n} \| = \frac{1}{n} \| T (x_n)\| \geq \frac{1}{n} \cdot n^3 \|x_n \| = n^2
\end{align*}
Thus we have shown that we can easily constract a sequence $(x_n)_{n \in \mathbb{N}}$ in $X$ that converges in norm to $0$ but for which we have 
\begin{align*}
\| T x_n \| \geq n^2 \text{ for all } n \in \mathbb{N}
\end{align*}
Let us therefore take such a sequence $(x_n)_{n \in \mathbb{N}}$ as above. We then have that $x_n \to 0$ in the strong (i.e. norm) sense and thus in particular $x_n \wto 0$. By our initial assumption this gives that $T(x_n) \wto T(0)=0$ in $Y$. This means that for all $f \in Y^*$ we have that 
\begin{align*}
f(T(x_n)) \to f(0)=0 
\end{align*}
Furthermore we can view $T(x_n)$ as an element $\Psi_n$ of $Y^{**}$ by using the canonical embedding $J_Y : Y \to Y^{**}$, setting $\Psi_n:= J_Y(T(x_n)) \in Y^{**}$. By definition of the canonical embedding we then have
\begin{align*}
\Psi_n(f) := J_Y(T(x_n))(f)=f(T(x_n))  \to 0 \text{ for all } f \in Y^*
\end{align*}
This yields that we have
\begin{align*}
\sup_{n \in \mathbb{N}} | \Psi_n(f)| \leq c_f
\end{align*}
Applying the Banach-Steinhaus Theorem to $Y^{**}$ we further get that there exists a constant $c \geq 0$ such that  
\begin{align*}
\sup_{n \in \mathbb{N}} \| \Psi_n\|_{Y^{**}} \leq c
\end{align*}
\newpage
But we recall that the canonical embedding $J_Y$ is an isometry between $Y$ and $Y^{**}$. Thus our derived bound yields
\begin{align*}
\sup_{n \in \mathbb{N}} \|\Psi_n\|_{Y^{**}}= \sup_{n \in \mathbb{N}} \| J_Y(T(x_n))\|_{Y^{**}} =\sup_{n \in \mathbb{N}} \| T(x_n)\|_Y \leq c
\end{align*}
Which is absurd because we have constructed our sequence to satisfy \begin{align*}
\|T (x_n)\|_Y \geq n^2 \to \infty \text{ as } n \to \infty
\end{align*}
arriving at a contradiction, we conclude that $T$ must be indeed bounded. 
\end{proof}
\begin{rem} We did not make use of the fact that $X$ is reflexive, maybe it is possible to show the above in a more elegant way when tightening the conditions for $X$ to be reflexive. 
\end{rem}
\newpage
\section{Eigenvalues of the Laplace operator}
Let $\Omega \subset \mathbb{R}^n$ be open, bounded and non-empty. We call $\lambda \in \mathbb{C}$ an eigenvalue of $- \Delta$ with zero boundary values, if there exists $u \in H_0^{1,2}(\Omega)$ such that $\|u\|_{L^2( \Omega)} =1$ and $- \Delta u = \lambda u$ in a weak sense, i.e. such that
\begin{align*}
\int_\Omega \nabla u \nabla \varphi dx = \lambda \int_\Omega u \varphi dx \text{ for all } \varphi \in H_0^{1,2}( \Omega)
\end{align*}
Then we call $u$ an eigenfunction to the eigenvalue $\lambda$. 
\\\\
\textbf{a)} Let $\lambda_1$ and $\lambda_2$ be eigenvalues of $- \Delta$ with eigenfunctions $u_1$ and $u_2$. Show that if $\lambda_1 \neq \lambda_2$, then $u_1$ and $u_2$ are orthogonal in $L^2$, i.e. $\langle u_1, u_2 \rangle_{L^2( \Omega)} = 0$ 
\begin{proof}
We have
\begin{align*}
\langle u_1, u_2 \rangle_{L^2} = \int_\Omega u_1(x) u_2(x) dx 
\end{align*}
Since $u_1, u_2$ are both eigenvalues of $- \Delta$ with eigenvalues $\lambda_1$ respectively $\lambda_2$ we can say that 
\begin{align*}
\lambda_1 \int_\Omega u_1(x) u_2(x)dx = \lambda_1 \langle u_1, u_2 \rangle_{L^2}
\end{align*}
using $\varphi=u_2 \in H_0^{1,2}( \Omega)$ as a test function in the above, similarly we have 
\begin{align*}
\lambda_2 \int_\Omega u_1(x) u_2(x)dx = \lambda_2 \langle u_1, u_2 \rangle_{L^2} 
\end{align*}
using $\varphi = u_1 \in H_0^{1,2}(\Omega)$ as a test function in the above definition. We thus obtain that
\begin{align*}
\underbrace{( \lambda_1 - \lambda_2)}_{ \neq 0} \langle u_1, u_2 \rangle_{L^2} = 0  \implies \langle u_1, u_2 \rangle_{L^2} =0 
\end{align*}
since by assumption $\lambda_1 \neq \lambda_2$. 
\end{proof}
\end{document}