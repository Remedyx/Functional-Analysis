\documentclass[11pt,a4paper]{article}
\usepackage[utf8]{inputenc}
\usepackage{amsmath}
\usepackage{amsfonts}
\usepackage{amssymb}
\usepackage{graphicx}
\usepackage{amsthm}
\title{Functional Analysis - The irreducible Minimum}
\date{}

\newtheorem{lem}{Lemma}[section]
\newtheorem{thm}{Theorem}[section]
\newtheorem{prop}{Proposition}[section]
\newtheorem{cor}{Corollary}[section]

\newtheorem{defn}{Definition}[section]
\newtheorem{exmp}{Example}[section]
\theoremstyle{definition}
\newtheorem{rem}{Remark}[section]

\begin{document}
\maketitle
\section{Structures}
\subsection{Topological spaces}
We will deal with various structures, in a sense, they build a certain hierachy. Topological spaces as a framework, then metric spaces, normed spaces, Banach spaces and conclusively Hilbert Spaces. 

\begin{defn} A topological space is a pair $(X, \tau)$, consisting of a set $X$ and a family $\tau \subset 2^X$ that we call a topology on $X$, such that 
\begin{enumerate}
\item $\emptyset, X \in \tau$.
\item Stable under countable intersection of open sets.
\item Stable under arbitrary union of open sets.
\end{enumerate}

\end{defn}
\begin{rem} Thanks to the laws of De Morgan we easily conclude from the definition that arbitrary intersections and finite unions of closed sets are again closed. 
\end{rem}
\begin{defn}
Let $(X, \tau)$ be a topological space and $A \subset X$ a subset. The closure $\overline{A}$ of $A$ is defined by 
\begin{align*}
\overline{A}= \bigcap \lbrace B \subset X : B \text{ is closed and } A \subset B \rbrace
\end{align*}
by definition, it is the smallest closed set that contains $A$. The set $A$ is called dense in $X$ if $\overline{A}=X$. The space $X$ is called separable, if it contains a countable dense set. Moreover, the interior of $A$ is defined through
\begin{align*}
A^\circ = \bigcup \lbrace B \subset A : B \text{ is open} \rbrace
\end{align*}
in other words, $A^\circ$ is the largest open set contained in $A$. Finally, the boundary of $A$ is defined as $\partial A = \overline{A} \setminus A^\circ$. 
\end{defn}
\begin{rem} Evidently, the closure and interior of a set depend on the choice of the topology. 
\end{rem}
\begin{defn} Let $(X, \tau)$ be a topological space, $x \in X$. A set $U \subset X$ is called an open neighbourhood of $x$ if $U \in \tau$ (i.e. $U$ is open) and $x \in U$. 
\end{defn}
\newpage
\begin{defn} Let $(X, \tau)$ be a topological space and let $(x_n)_{n \in \mathbb{N}}$ be a sequence in $X$. We say that $x_n$ converges to $x \in X$ as $n \to \infty$, written $x_n \to X$, if \begin{align*}
\text{For every open neighbourhood $U$ of $x$, there exists $n_0 \in \mathbb{N}: x_n \in U \ \forall n \geq n_0.$}
\end{align*}
\end{defn}
\begin{defn} Let $(X, \tau), (Y , \mathcal{S})$ be two topological spaces. A function $f: X \to Y$ is called continuous if 
\begin{align*}
f^{-1}(V) \in \tau, \ \forall V \in \mathcal{S}.
\end{align*}
I.e., if the pre-image of every open set in $Y$ is an open set in $X$. 
\end{defn}
\begin{rem} In metric spaces, the notion of convergence completely characterizes the topology. This is however not necessarily true in topological spaces.
\end{rem}
\begin{defn} A topological space $(X, \tau)$ is called Hausdorff if  
\begin{align*}
\forall x,y \in X, x \neq y \implies \exists U_x, U_y \in \tau \text{ with } x \in U_x, y \in U_y \text{ and } U_x \cap U_y = \emptyset
\end{align*}
\end{defn}
\begin{defn} A topological space $(X, \tau)$ is called compact if it is Hausdorff and if for every \begin{align*}
(U_\lambda)_{\lambda \in \Lambda} \text{ family in } \tau \text{ with } \bigcup_{ \lambda \in \Lambda} U_\lambda = X \\ \implies \exists n \in \mathbb{N} \text{ and  } \lambda_1, \dots , \lambda_n \in \Lambda : \bigcup_{j=1}^n U_{\lambda_j}=X 
\end{align*}
i.e. if for every open covering of $X$ there exists a finite sub-covering. 
\end{defn}
\begin{thm} Let $(X, \tau)$ be a compact space and $(x_n)_{n \in \mathbb{N}}$ be a sequence in $X$. Then there exists at least one accumulation point of $x_n$ on $X$. 
\end{thm}
\begin{thm}[Lemma von Urysohn] Let $(X, \tau)$ be a compact space and $A,B \subset X$ disjoint, non-empty, closed subsets of $X$. Then there exists a continuous function $g: X \to [0,1]$ with $g(A)=0$ and $g(B)=1$. 
\end{thm}
\begin{rem} The Lemma of Urysohn is important because it helps us avoid some of the pathological situations (like only constant functions being continuous). In particular, the theorem implies that on compact spaces the topology is large enough for us to do some meaningful analysis on. 
\end{rem}
\newpage
\begin{defn} Let $K$ be a compact space. We define
\begin{align*}
C_\mathbb{K}(K):= \lbrace f: K \to \mathbb{K} \text{ cotninuous}\rbrace
\end{align*}
\end{defn}
\noindent As a consequence of Urysohn's Lemma we can show that $C_\mathbb{K}(K)$ separates the points of $K$. 
\begin{cor} Let $K$ be a compact space. Then $C_\mathbb{K}(K)$ separates the points of $K$. In other words, for every $x,y \in K$ with $x \neq y$, there exists $f \in C_\mathbb{K}(K)$ such that $f(x) \neq f(y)$.
\end{cor}
\begin{proof}
Since $K$ is Hausdorff, for every $x \neq y$ we can find two open neighbourhoods $U_x, U_y$ of $x$ and $y$ respectively with $U_x \cap U_y = \emptyset$. We can further find closed sets $A,B$ with $A \subset U_x, B \subset U_y$ with $x \in A$ and $y \in B$. In particular we have $A \cap B = \emptyset$. Urysohn's Lemma then gives that there exists $f \in C_\mathbb{K}(K)$ with $f(x)=0$ and $f(y)=1$. 
\end{proof}
\subsection{Metric Spaces}
\begin{defn} A metric space is a pair $(X,d)$ with $X$ being an arbitrary set and a map $d: X \times X \to [0, \infty)$ called a metric on $X$, with the following properties
\begin{enumerate}
\item $d(x,y)=0$ if and only if $x=y$.
\item $d(x,y)=d(y,x)$.
\item ($\Delta$-inequality) $d(x,y) \leq d(x,z) + d(z,y)$ for all $x,y,z \in X$. 
\end{enumerate}
\end{defn}
\begin{rem} \
\begin{enumerate}
\item Every metric space $(X,d)$ is a topological space $(X, \tau_d)$ with topology $\tau_d$ inducedby the metric. The topology $\tau_d$ is defined by the condition that $A \in \tau_d$ if and only if for all $x \in A$, there exists $\epsilon >0$ with $B_\epsilon(x)= \lbrace y \in X : d(x,y) < \epsilon \rbrace \subset A$. 
\item In contrast with general topological spaces, on metric spaces the notion of convergence characterizes the topology. That is we have the convenient characterizations
\begin{enumerate}
\item A set $A\subset X$ is closed if and only if for every sequence $x_n$ in $A$ with $x_n \to x$ in $X$, we have that $x \in A$. 
\item $\overline{A}= \lbrace x \in X : \exists (x_n)_{n \in \mathbb{N}} \text{ sequence in $A$ with } x_n \to x \rbrace $. 
\item $A \subset X$ is dense if and only if $\forall x \in X, \exists (x_n)_{n \in \mathbb{N}}$ sequence in $A$ with $x_n \to x$. 
\item The function $f: X \to Y$ between the two metric spaces $(X,d_1), (Y,d_2)$ is continuous at the point $x \in X$ if and only if for every sequence $(x_n)_{n \in \mathbb{N}}$ in $X$ with $x_n \to x$ we have $f(x_n) \to f(x)$. 
\end{enumerate}
\end{enumerate}
\end{rem}
\newpage
\begin{defn} Let $(X,d)$ be a metric space. A sequence $(x_n)_{n \in \mathbb{N}}$ in $X$ is called a Cauchy sequence (or is said to have the Cauchy property) if $d(x_n,x_m) \to 0$ as $n,m \to \infty$. Every convergent sequence is indeed Cauchy. The metric space $(X,d)$ is called complete, if every Cauchy sequence is convergent. 
\end{defn}
\subsection{Normed spaces}
\begin{defn} A normed space is a pair $(X, \| \cdot \|)$, consisting of a vector space $V$ over a field $\mathbb{K}$ ($\mathbb{R}$ or $\mathbb{C}$) and a map $\| \cdot  \|: V \to [0, \infty)$ called a norm on $V$ with the following properties
\begin{enumerate}
\item $\|x\|=0$ if and only if $x=0$.
\item (Homogenity) $\| \lambda x \| = | \lambda | \|x \|$ for all $\lambda \in \mathbb{K}, x \in V$.
\item ($\Delta$-inequality) $\|x + y\| \leq \|x\|  + \|y\|$ for all $x,y \in V$. 
\end{enumerate}
\end{defn}
\begin{rem} Every norm induces the metric $d(x,y)= \|x-y\|$, hence every normed space is also a metric space and therefore also a topological space. 
\end{rem}
\begin{defn} A normed space $(X, \| \cdot \|)$ is called complete, if $X$, equipped with the induced metric $d(x,y)= \|x-y\|$, is a complete metric space. A complete normed space is called a Banach space. 
\end{defn}
\noindent Completeness is very important for analysis. It is not a coincidence that we always do analysis on $\mathbb{R}$ instead of $\mathbb{Q}$. For this reason, it is useful to have a general recipe to complete normed spaces. 
\begin{defn} Let $(X, \| \cdot \|)$ be a normed space. A completion of $(X, \| \cdot \|)$ is a $3$-tuple $(Y, \| \cdot \|_Y , \phi)$ consisting of a Banach space $(Y, \| \cdot \|_Y)$ and an isometric linear map $\phi :X \to Y$, with $\overline{\phi(X)}=Y$.
\end{defn}
\begin{thm} Every normed space $(X, \| \cdot \|)$ has a completion, which is unique, up to linear isometric isomorphisms. 
\end{thm}
\begin{proof}
The proof is constructive, we give a sketch.  Let $\mathcal{C}_X$ denote the set of all Cauchy sequence on $X$. We can easily give this space the structure of a vector space over $\mathbb{K}$. Next we define the linear subspace $\mathcal{N}_X \subset \mathcal{C}_X$ consisting of all null-sequence on $X$, i.e.
\begin{align*}
\mathcal{N}_X := \lbrace x = (x_n)_{n \in \mathbb{N}} \in \mathcal{C}_X : x_n \to 0 \rbrace
\end{align*}
Moreover we define $Y:= \mathcal{C}_X/ \mathcal{N}_X$ as the quotient space of $\mathcal{C}_X$ w.r.t. the equivalence relation defined by $x \sim y : \iff x-y \in \mathcal{N}_X$. In other words, in $Y$, we identify Cauchy sequences whose difference converges to zero. $Y$ is also a vector space over $\mathbb{K}$. 
\newpage
Next we want to introduce a norm on $Y$. To this end, we define the function $p: \mathcal{C}_X \to [0, \infty)$ through 
\begin{align*}
p(x)= \lim_{n \to \infty} \| x_n \|
\end{align*}
Thanks to the reverse triangle inequality we've got
\begin{align*}
| \|x_k\| - \|x_l\| | \leq \|x_k - x_l\| \to 0 \text{ as } k,l \to \infty
\end{align*}
which shows that $(\|x_n\|)_{n \in \mathbb{N}}$ is a Cauchy sequence in $\mathbb{R}$ whenever $(x_n)_{n \in \mathbb{N}}$ is a sequence in $\mathcal{C}_X$. Thus the limit above is well-defined and finite. We now set 
\begin{align*}
\|[x]\|_Y := p(x)= \lim_{n \to \infty} \|x_n\|
\end{align*}
We then can verify that $\| \cdot \|_Y$ is indeed a norm on $Y$. \\
\\
Next we define the map $\phi : X \to Y$ by $\phi (z)=[(z,z, \dots )]$, i.e. $\phi(z)$ denotes the equivalence class of all sequences on $X$ that converge to $z$ in the limit. This map is clearly linear and since 
\begin{align*}
\| \phi(z)\|_Y= \|z\|_X
\end{align*}
it defines an isometry. We now claim that $(Y, \| \cdot \|_Y, \phi)$ is a completion of $(X, \| \cdot \|_X)$. In order to show that we can show that $( Y, \| \cdot \|_Y)$ is always complete and that $\phi(X)$ is dense in $Y$. That is we want to show that for all $[x] \in Y$, we can find $\tilde{x} \in X$ with $\| \phi(\tilde{x}) - [x]\|_Y < \epsilon$. We start with the density and then use this to show that the space is complete. 
\\\\
Finally we can show that the uniqueness of the completion is up to isometric isomorphisms. 
\end{proof}
\begin{rem} The completion of $\mathbb{Q}$ is $\mathbb{R}$. For $\Omega \subset \mathbb{R}^n$, the completion of the space $C( \overline{\Omega})$ of continuous functions on $\overline{\Omega}$ is $L^p(\Omega)$.
\end{rem}
\subsection{Hilbert Spaces}
\begin{defn} Let $H$ be a vector space over the field $\mathbb{K}$. A scalar product (or an inner product) on $H$ is a map $(\cdot, \cdot) : H \times H \to \mathbb{K}$ with the properties 
\begin{enumerate}
\item $(z,x+ \lambda y)= (z,x) + \lambda(z,y)$ for all $x,y,z \in H, \lambda \in \mathbb{K}$.
\item $(x,y)= \overline{(y,x)}$.
\item $(x,x)>0$ for all $x \neq 0$.
\end{enumerate}
A pair $(H, ( \cdot, \cdot))$ consisting of a vector space $H$ over $\mathbb{K}$ and a scalar product $( \cdot, \cdot)$ is called a pre-Hilbert space. 
\end{defn}
\begin{rem} We defiend the scalar product to be linear in its second argument and anti-linear in its first argument. 
\end{rem}
\newpage
\begin{lem}[Cauchy Schwarz Inequality] Let $(H,( \cdot, \cdot))$ be a pre-Hilbert space. Then 
\begin{align*}
|(x,y)|^2 \leq (x,x) (y,y). 
\end{align*}
\end{lem}
\begin{rem} The Cauchy-Schwarz inequality allows us to use the scalar product to define a norm on every pre-Hilbert space. 
\end{rem}
\begin{cor} Let $(H, ( \cdot, \cdot))$ be a pre-Hilbert space. Then 
\begin{align*}
\|x\|:= \sqrt{(x,x)}
\end{align*}
defines a norm on $H$. 
\end{cor}
\begin{rem} The triangle inequality follows from the Cauchy-Schwarz inequality. The remaining properties follow from the properties of scalar products. 
\end{rem}
\begin{defn} A pre-Hilbert space is called a Hilbert space if $H$, equipped with the norm $\|x \| = \sqrt{(x,x)}$ induced by the scalar product, is a Banach space (i.e. if $H$ is complete). 
\end{defn}
\begin{rem} \
\begin{enumerate} 
\item Every pre-Hilbert space can be completed into a Hilbert space. 
\item Every Hilbert space is a metric space and therefore a topological space. However, clearly, not every Banach space is a Hilbert space, simply because not all norms on a vector space can be induced by a scalar product. 
\end{enumerate}
\end{rem}
\begin{thm} Let $(H, ( \cdot, \cdot))$ be a Hilbert space, $K \subset H$ a closed convex set in $H$ and $x_0 \in H$. Then there exists a unique $y \in K$ such that 
\begin{align*}
\| x_0-y\| = \text{dist}(x_0,K):= \inf_{x \in K} \| x_0-x\|
\end{align*}
\end{thm}
\noindent As an application of this theorem, we show that every Hilbert space $H$  can be decomposed in the direct sum of an arbitrary closed subspace and of its orthogonal complement. 
\begin{thm} Let $(H, ( \cdot , \cdot))$ be a Hilbert space and $M \subset H$ a closed linear subspace. Then the orthogonal complement $M^\perp$ of $M$, defined through
\begin{align*}
M^\perp := \lbrace x \in H : (x,m)=0, \text{ for all } m \in M \rbrace 
\end{align*}
is also a linear closed subspace of $H$ and $H= M \oplus M^\perp$, meaning that $H=M+M^\perp$ and $M \cap M^\perp = \lbrace 0 \rbrace$. 
\end{thm}
\newpage
\begin{proof}
Clearly $M^\perp$ is linear. In order to see that it is closed take $x_n$ to be a sequence in $M^\perp$ such that $x_n \to x$ in $H$. Then we have 
\begin{align*}
(x,m)= \lim_{n \to \infty} (x_n,m)=0
\end{align*}
because thanks to the Cauchy-Schwarz inequality we have 
\begin{align*}
|(x-x_n, m)| \leq \|x-x_n\| \|m\| \to 0 \text{ as } n \to \infty 
\end{align*}
in particular $|(x-x_n,m)|=|(x,m)-(x_n,m)| \to 0$ and it follows that $x \in M^\perp$, i.e. $M^\perp$ is closed. 
\\\\
Moreover, the fact that $M \cap M^\perp = \lbrace 0 \rbrace$ follows, because $(x,x)=0$ implies that $x=0$. Thus it only rmeains to show that $M+M^\perp = H$. To this end, we fix $x \in H$. Since $M \subset H$ is a closed linear subspace (and therefore in particular convex) we can apply the previous theorem to find $z \in M$ such that dist$(x,M)=\|x-z\|$.
\\\\
We now claim that $x-z  \in M^\perp$, which gives that $x = z + (x-z)$ is the desired decomposition of $H$ into $M$ and $M^\perp$. Lets assume for contradiction that $(x-z) \notin M^\perp$. Then there exists $\alpha \in M$ with $(x-z, \alpha )>0$. For $t \in [-1,1]$ let $z_t = z + t \alpha$. Then we have $z_t \in M$ for all $t$ and
\begin{align*}
\|x-z_t\|^2 = \|x-z\|^2 + t^2\| \alpha\|^2 -2t(x-z, \alpha) < \|x-z\|^2 = \text{dist}(x,M)
\end{align*}
for $t >0$ small enough. But this condradicts the definition of dist$(x,M)$. 
\end{proof}
\begin{defn} An orthonormal system in $(H, ( \cdot, \cdot))$ is a family $(x_\alpha)_{\alpha \in A} \subset H$ for an arbitrary index-set $A$ with $(x_\alpha, x_\beta)= \delta_{\alpha, \beta}$. In the case when $A = \mathbb{N}$, we also call the orthonormal system an orthonormal sequence. 
\end{defn}
\begin{lem} Let $H$ be a Hilbert space, $(x_n)_{n \in \mathbb{N}}$ an orthonormal system (orthonormal sequence) and $(\alpha_n)_{n \in \mathbb{N}}$ a sequence in $\mathbb{K}$. Then we have
\begin{enumerate}
\item $\sum_{k=1}^\infty \alpha_k x_k$ converges if and only if $\sum_{k=1}^\infty | \alpha_k|^2$ converges. 
\item $\| \sum_{k=1}^n \alpha_k x_k\|^2 = \sum_{k=1}^n |\alpha_k|^2$.
\item If $\sum_{k=1}^\infty \alpha_k x_k$ converges, then the limit is independent of the order of the terms. 
\end{enumerate}
\end{lem}
\begin{lem} Let $(H, ( \cdot , \cdot))$ be a Hilbert space, $A$ an arbitrary set an $(x_\alpha)_{\alpha \in A}$ an orthonormal system in $H$. Then $\sum_{ \alpha \in A} ( x_\alpha ,x) x_\alpha$ converges for every $x \in H$. Moreover, the linear map $\phi : H \to H$ defined through $\phi(x) = \sum_{ \alpha \in A} (x_\alpha, x) x_\alpha$ is the continuous projection onto
\begin{align*}
M:= \overline{\text{span}\lbrace x_\alpha : \alpha \in A \rbrace } \text{ along its orthogonal complement } M^\perp
\end{align*}
In particular for $x \in M$, we find that $x = \sum_{ \alpha \in A} (x_\alpha, x) x_\alpha$. 
\end{lem}
\newpage
\begin{rem} In particular, if $M = \text{span} \lbrace x_\alpha :  \alpha \in A \rbrace$ is dense in $H$, i.e. if $\overline{M}=H$, then gives us the previous Lemma a representation for every vector $x \in H$. In this case, we say that $(x_\alpha)_{ \alpha \in A}$ is a Hilbert space basis. 
\end{rem}
\begin{defn} Let $H$ be a Hilbert space. A Hilbert space basis is an orthonormal system $(x_\alpha)_{ \alpha \in A}$ with \begin{align*}
\overline{\text{span} \lbrace x_\alpha : \alpha \in A \rbrace }=H
\end{align*}
\end{defn}
\begin{thm}[Characterizations of Hilbert space bases] Let $H$ be a Hilbert space, and $(x_\alpha)_{ \alpha \in A}$ an orthonormal system. Then the following statements are equivalent:
\begin{enumerate}
\item $(x_\alpha)_{ \alpha \in A}$ is a Hilbert space basis.
\item $x= \sum_{ \alpha \in A} (x_\alpha, x) x_\alpha$, for all $x \in H$.
\item $\|x\|^2 = \sum_{ \alpha} |(x_\alpha, x)|^2$ for all $x \in H$. 
\item $(x_\alpha, x)=0$ for all $\alpha \in A$ implies that $x=0$.
\item $(x_\alpha)_{\alpha \in A}$ is a maximal orthonormal system in the sense of inclusions. 
\end{enumerate}
\end{thm}
\begin{rem} Using the maximality property, i.e. point 5 in the theorem above, it follows easily from the Lemma of Zorn that every pre-Hilbert space has a Hilbert space basis. In particular, it follows that every separable Hilbert space admits a countable orthonormal basis. The observation that separable Hilbert spaces have countable orthonormal bases can be used to identify separable Hilbert spaces with the sequence space $\ell^2( \mathbb{K})$. 
\end{rem}
\begin{thm} Let $H$ be an infinite dimensional separable Hilbert space over $\mathbb{K}$. Then there exists a linear Isomorphism $\phi: H \to \ell^2( \mathbb{K})$ with 
\begin{align*}
( \phi(x), \phi(y))_{ \ell^2} =(x,y)_H
\end{align*}
for all $x,y \in H$. In particular the isomorphism is isometric. 
\end{thm}
\end{document}