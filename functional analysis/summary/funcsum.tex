\documentclass[11pt,a4paper]{article}
\usepackage[utf8]{inputenc}
\usepackage{amsmath}
\usepackage{amsfonts}
\usepackage{amssymb}
\usepackage{graphicx}
\usepackage{amsthm}
\title{Functional Analysis - The irreducible Minimum}
\date{}

\newtheorem{lem}{Lemma}[section]
\newtheorem{thm}{Theorem}[section]
\newtheorem{prop}{Proposition}[section]
\newtheorem{cor}{Corollary}[section]

\newtheorem{defn}{Definition}[section]
\newtheorem{exmp}{Example}[section]
\theoremstyle{definition}
\newtheorem{rem}{Remark}[section]

\newcommand{\wto}{\rightharpoonup}

\begin{document}
\maketitle
\section{Structures}
\subsection{Topological spaces}
We will deal with various structures, in a sense, they build a certain hierachy. Topological spaces as a framework, then metric spaces, normed spaces, Banach spaces and conclusively Hilbert Spaces. 

\begin{defn} A topological space is a pair $(X, \tau)$, consisting of a set $X$ and a family $\tau \subset 2^X$ that we call a topology on $X$, such that 
\begin{enumerate}
\item $\emptyset, X \in \tau$.
\item Stable under countable intersection of open sets.
\item Stable under arbitrary union of open sets.
\end{enumerate}

\end{defn}
\begin{rem} Thanks to the laws of De Morgan we easily conclude from the definition that arbitrary intersections and finite unions of closed sets are again closed. 
\end{rem}
\begin{defn}
Let $(X, \tau)$ be a topological space and $A \subset X$ a subset. The closure $\overline{A}$ of $A$ is defined by 
\begin{align*}
\overline{A}= \bigcap \lbrace B \subset X : B \text{ is closed and } A \subset B \rbrace
\end{align*}
by definition, it is the smallest closed set that contains $A$. The set $A$ is called dense in $X$ if $\overline{A}=X$. The space $X$ is called separable, if it contains a countable dense set. Moreover, the interior of $A$ is defined through
\begin{align*}
A^\circ = \bigcup \lbrace B \subset A : B \text{ is open} \rbrace
\end{align*}
in other words, $A^\circ$ is the largest open set contained in $A$. Finally, the boundary of $A$ is defined as $\partial A = \overline{A} \setminus A^\circ$. 
\end{defn}
\begin{rem} Evidently, the closure and interior of a set depend on the choice of the topology. 
\end{rem}
\begin{defn} Let $(X, \tau)$ be a topological space, $x \in X$. A set $U \subset X$ is called an open neighbourhood of $x$ if $U \in \tau$ (i.e. $U$ is open) and $x \in U$. 
\end{defn}
\newpage
\begin{defn} Let $(X, \tau)$ be a topological space and let $(x_n)_{n \in \mathbb{N}}$ be a sequence in $X$. We say that $x_n$ converges to $x \in X$ as $n \to \infty$, written $x_n \to X$, if \begin{align*}
\text{For every open neighbourhood $U$ of $x$, there exists $n_0 \in \mathbb{N}: x_n \in U \ \forall n \geq n_0.$}
\end{align*}
\end{defn}
\begin{defn} Let $(X, \tau), (Y , \mathcal{S})$ be two topological spaces. A function $f: X \to Y$ is called continuous if 
\begin{align*}
f^{-1}(V) \in \tau, \ \forall V \in \mathcal{S}.
\end{align*}
I.e., if the pre-image of every open set in $Y$ is an open set in $X$. 
\end{defn}
\begin{rem} In metric spaces, the notion of convergence completely characterizes the topology. This is however not necessarily true in topological spaces.
\end{rem}
\begin{defn} A topological space $(X, \tau)$ is called Hausdorff if  
\begin{align*}
\forall x,y \in X, x \neq y \implies \exists U_x, U_y \in \tau \text{ with } x \in U_x, y \in U_y \text{ and } U_x \cap U_y = \emptyset
\end{align*}
\end{defn}
\begin{defn} A topological space $(X, \tau)$ is called compact if it is Hausdorff and if for every \begin{align*}
(U_\lambda)_{\lambda \in \Lambda} \text{ family in } \tau \text{ with } \bigcup_{ \lambda \in \Lambda} U_\lambda = X \\ \implies \exists n \in \mathbb{N} \text{ and  } \lambda_1, \dots , \lambda_n \in \Lambda : \bigcup_{j=1}^n U_{\lambda_j}=X 
\end{align*}
i.e. if for every open covering of $X$ there exists a finite sub-covering. 
\end{defn}
\begin{thm} Let $(X, \tau)$ be a compact space and $(x_n)_{n \in \mathbb{N}}$ be a sequence in $X$. Then there exists at least one accumulation point of $x_n$ on $X$. 
\end{thm}
\begin{thm}[Lemma von Urysohn] Let $(X, \tau)$ be a compact space and $A,B \subset X$ disjoint, non-empty, closed subsets of $X$. Then there exists a continuous function $g: X \to [0,1]$ with $g(A)=0$ and $g(B)=1$. 
\end{thm}
\begin{rem} The Lemma of Urysohn is important because it helps us avoid some of the pathological situations (like only constant functions being continuous). In particular, the theorem implies that on compact spaces the topology is large enough for us to do some meaningful analysis on. 
\end{rem}
\newpage
\begin{defn} Let $K$ be a compact space. We define
\begin{align*}
C_\mathbb{K}(K):= \lbrace f: K \to \mathbb{K} \text{ continuous}\rbrace
\end{align*}
\end{defn}
\noindent As a consequence of Urysohn's Lemma we can show that $C_\mathbb{K}(K)$ separates the points of $K$. 
\begin{cor} Let $K$ be a compact space. Then $C_\mathbb{K}(K)$ separates the points of $K$. In other words, for every $x,y \in K$ with $x \neq y$, there exists $f \in C_\mathbb{K}(K)$ such that $f(x) \neq f(y)$.
\end{cor}
\begin{proof}
Since $K$ is Hausdorff, for every $x \neq y$ we can find two open neighbourhoods $U_x, U_y$ of $x$ and $y$ respectively with $U_x \cap U_y = \emptyset$. We can further find closed sets $A,B$ with $A \subset U_x, B \subset U_y$ with $x \in A$ and $y \in B$. In particular we have $A \cap B = \emptyset$. Urysohn's Lemma then gives that there exists $f \in C_\mathbb{K}(K)$ with $f(x)=0$ and $f(y)=1$. 
\end{proof}
\subsection{Metric Spaces}
\begin{defn} A metric space is a pair $(X,d)$ with $X$ being an arbitrary set and a map $d: X \times X \to [0, \infty)$ called a metric on $X$, with the following properties
\begin{enumerate}
\item $d(x,y)=0$ if and only if $x=y$.
\item $d(x,y)=d(y,x)$.
\item ($\Delta$-inequality) $d(x,y) \leq d(x,z) + d(z,y)$ for all $x,y,z \in X$. 
\end{enumerate}
\end{defn}
\begin{rem} \
\begin{enumerate}
\item Every metric space $(X,d)$ is a topological space $(X, \tau_d)$ with topology $\tau_d$ inducedby the metric. The topology $\tau_d$ is defined by the condition that $A \in \tau_d$ if and only if for all $x \in A$, there exists $\epsilon >0$ with $B_\epsilon(x)= \lbrace y \in X : d(x,y) < \epsilon \rbrace \subset A$. 
\item In contrast with general topological spaces, on metric spaces the notion of convergence characterizes the topology. That is we have the convenient characterizations
\begin{enumerate}
\item A set $A\subset X$ is closed if and only if for every sequence $x_n$ in $A$ with $x_n \to x$ in $X$, we have that $x \in A$. 
\item $\overline{A}= \lbrace x \in X : \exists (x_n)_{n \in \mathbb{N}} \text{ sequence in $A$ with } x_n \to x \rbrace $. 
\item $A \subset X$ is dense if and only if $\forall x \in X, \exists (x_n)_{n \in \mathbb{N}}$ sequence in $A$ with $x_n \to x$. 
\item The function $f: X \to Y$ between the two metric spaces $(X,d_1), (Y,d_2)$ is continuous at the point $x \in X$ if and only if for every sequence $(x_n)_{n \in \mathbb{N}}$ in $X$ with $x_n \to x$ we have $f(x_n) \to f(x)$. 
\end{enumerate}
\end{enumerate}
\end{rem}
\newpage
\begin{defn} Let $(X,d)$ be a metric space. A sequence $(x_n)_{n \in \mathbb{N}}$ in $X$ is called a Cauchy sequence (or is said to have the Cauchy property) if $d(x_n,x_m) \to 0$ as $n,m \to \infty$. Every convergent sequence is indeed Cauchy. The metric space $(X,d)$ is called complete, if every Cauchy sequence is convergent. 
\end{defn}
\subsection{Normed spaces}
\begin{defn} A normed space is a pair $(X, \| \cdot \|)$, consisting of a vector space $V$ over a field $\mathbb{K}$ ($\mathbb{R}$ or $\mathbb{C}$) and a map $\| \cdot  \|: V \to [0, \infty)$ called a norm on $V$ with the following properties
\begin{enumerate}
\item $\|x\|=0$ if and only if $x=0$.
\item (Homogenity) $\| \lambda x \| = | \lambda | \|x \|$ for all $\lambda \in \mathbb{K}, x \in V$.
\item ($\Delta$-inequality) $\|x + y\| \leq \|x\|  + \|y\|$ for all $x,y \in V$. 
\end{enumerate}
\end{defn}
\begin{rem} Every norm induces the metric $d(x,y)= \|x-y\|$, hence every normed space is also a metric space and therefore also a topological space. 
\end{rem}
\begin{defn} A normed space $(X, \| \cdot \|)$ is called complete, if $X$, equipped with the induced metric $d(x,y)= \|x-y\|$, is a complete metric space. A complete normed space is called a Banach space. 
\end{defn}
\noindent Completeness is very important for analysis. It is not a coincidence that we always do analysis on $\mathbb{R}$ instead of $\mathbb{Q}$. For this reason, it is useful to have a general recipe to complete normed spaces. 
\begin{defn} Let $(X, \| \cdot \|)$ be a normed space. A completion of $(X, \| \cdot \|)$ is a $3$-tuple $(Y, \| \cdot \|_Y , \phi)$ consisting of a Banach space $(Y, \| \cdot \|_Y)$ and an isometric linear map $\phi :X \to Y$, with $\overline{\phi(X)}=Y$.
\end{defn}
\begin{thm} Every normed space $(X, \| \cdot \|)$ has a completion, which is unique, up to linear isometric isomorphisms. 
\end{thm}
\begin{proof}
The proof is constructive, we give a sketch.  Let $\mathcal{C}_X$ denote the set of all Cauchy sequence on $X$. We can easily give this space the structure of a vector space over $\mathbb{K}$. Next we define the linear subspace $\mathcal{N}_X \subset \mathcal{C}_X$ consisting of all null-sequence on $X$, i.e.
\begin{align*}
\mathcal{N}_X := \lbrace x = (x_n)_{n \in \mathbb{N}} \in \mathcal{C}_X : x_n \to 0 \rbrace
\end{align*}
Moreover we define $Y:= \mathcal{C}_X/ \mathcal{N}_X$ as the quotient space of $\mathcal{C}_X$ w.r.t. the equivalence relation defined by $x \sim y : \iff x-y \in \mathcal{N}_X$. In other words, in $Y$, we identify Cauchy sequences whose difference converges to zero. $Y$ is also a vector space over $\mathbb{K}$. 
\newpage
Next we want to introduce a norm on $Y$. To this end, we define the function $p: \mathcal{C}_X \to [0, \infty)$ through 
\begin{align*}
p(x)= \lim_{n \to \infty} \| x_n \|
\end{align*}
Thanks to the reverse triangle inequality we've got
\begin{align*}
| \|x_k\| - \|x_l\| | \leq \|x_k - x_l\| \to 0 \text{ as } k,l \to \infty
\end{align*}
which shows that $(\|x_n\|)_{n \in \mathbb{N}}$ is a Cauchy sequence in $\mathbb{R}$ whenever $(x_n)_{n \in \mathbb{N}}$ is a sequence in $\mathcal{C}_X$. Thus the limit above is well-defined and finite. We now set 
\begin{align*}
\|[x]\|_Y := p(x)= \lim_{n \to \infty} \|x_n\|
\end{align*}
We then can verify that $\| \cdot \|_Y$ is indeed a norm on $Y$. \\
\\
Next we define the map $\phi : X \to Y$ by $\phi (z)=[(z,z, \dots )]$, i.e. $\phi(z)$ denotes the equivalence class of all sequences on $X$ that converge to $z$ in the limit. This map is clearly linear and since 
\begin{align*}
\| \phi(z)\|_Y= \|z\|_X
\end{align*}
it defines an isometry. We now claim that $(Y, \| \cdot \|_Y, \phi)$ is a completion of $(X, \| \cdot \|_X)$. In order to show that we can show that $( Y, \| \cdot \|_Y)$ is always complete and that $\phi(X)$ is dense in $Y$. That is we want to show that for all $[x] \in Y$, we can find $\tilde{x} \in X$ with $\| \phi(\tilde{x}) - [x]\|_Y < \epsilon$. We start with the density and then use this to show that the space is complete. 
\\\\
Finally we can show that the uniqueness of the completion is up to isometric isomorphisms. 
\end{proof}
\begin{rem} The completion of $\mathbb{Q}$ is $\mathbb{R}$. For $\Omega \subset \mathbb{R}^n$, the completion of the space $C( \overline{\Omega})$ of continuous functions on $\overline{\Omega}$ is $L^p(\Omega)$.
\end{rem}
\subsection{Hilbert Spaces}
\begin{defn} Let $H$ be a vector space over the field $\mathbb{K}$. A scalar product (or an inner product) on $H$ is a map $(\cdot, \cdot) : H \times H \to \mathbb{K}$ with the properties 
\begin{enumerate}
\item $(z,x+ \lambda y)= (z,x) + \lambda(z,y)$ for all $x,y,z \in H, \lambda \in \mathbb{K}$.
\item $(x,y)= \overline{(y,x)}$.
\item $(x,x)>0$ for all $x \neq 0$.
\end{enumerate}
A pair $(H, ( \cdot, \cdot))$ consisting of a vector space $H$ over $\mathbb{K}$ and a scalar product $( \cdot, \cdot)$ is called a pre-Hilbert space. 
\end{defn}
\begin{rem} We defiend the scalar product to be linear in its second argument and anti-linear in its first argument. 
\end{rem}
\newpage
\begin{lem}[Cauchy Schwarz Inequality] Let $(H,( \cdot, \cdot))$ be a pre-Hilbert space. Then 
\begin{align*}
|(x,y)|^2 \leq (x,x) (y,y). 
\end{align*}
\end{lem}
\begin{rem} The Cauchy-Schwarz inequality allows us to use the scalar product to define a norm on every pre-Hilbert space. 
\end{rem}
\begin{cor} Let $(H, ( \cdot, \cdot))$ be a pre-Hilbert space. Then 
\begin{align*}
\|x\|:= \sqrt{(x,x)}
\end{align*}
defines a norm on $H$. 
\end{cor}
\begin{rem} The triangle inequality follows from the Cauchy-Schwarz inequality. The remaining properties follow from the properties of scalar products. 
\end{rem}
\begin{defn} A pre-Hilbert space is called a Hilbert space if $H$, equipped with the norm $\|x \| = \sqrt{(x,x)}$ induced by the scalar product, is a Banach space (i.e. if $H$ is complete). 
\end{defn}
\begin{rem} \
\begin{enumerate} 
\item Every pre-Hilbert space can be completed into a Hilbert space. 
\item Every Hilbert space is a metric space and therefore a topological space. However, clearly, not every Banach space is a Hilbert space, simply because not all norms on a vector space can be induced by a scalar product. 
\end{enumerate}
\end{rem}
\begin{thm} Let $(H, ( \cdot, \cdot))$ be a Hilbert space, $K \subset H$ a closed convex set in $H$ and $x_0 \in H$. Then there exists a unique $y \in K$ such that 
\begin{align*}
\| x_0-y\| = \text{dist}(x_0,K):= \inf_{x \in K} \| x_0-x\|
\end{align*}
\end{thm}
\noindent As an application of this theorem, we show that every Hilbert space $H$  can be decomposed in the direct sum of an arbitrary closed subspace and of its orthogonal complement. 
\begin{thm} Let $(H, ( \cdot , \cdot))$ be a Hilbert space and $M \subset H$ a closed linear subspace. Then the orthogonal complement $M^\perp$ of $M$, defined through
\begin{align*}
M^\perp := \lbrace x \in H : (x,m)=0, \text{ for all } m \in M \rbrace 
\end{align*}
is also a linear closed subspace of $H$ and $H= M \oplus M^\perp$, meaning that $H=M+M^\perp$ and $M \cap M^\perp = \lbrace 0 \rbrace$. 
\end{thm}
\newpage
\begin{proof}
Clearly $M^\perp$ is linear. In order to see that it is closed take $x_n$ to be a sequence in $M^\perp$ such that $x_n \to x$ in $H$. Then we have 
\begin{align*}
(x,m)= \lim_{n \to \infty} (x_n,m)=0
\end{align*}
because thanks to the Cauchy-Schwarz inequality we have 
\begin{align*}
|(x-x_n, m)| \leq \|x-x_n\| \|m\| \to 0 \text{ as } n \to \infty 
\end{align*}
in particular $|(x-x_n,m)|=|(x,m)-(x_n,m)| \to 0$ and it follows that $x \in M^\perp$, i.e. $M^\perp$ is closed. 
\\\\
Moreover, the fact that $M \cap M^\perp = \lbrace 0 \rbrace$ follows, because $(x,x)=0$ implies that $x=0$. Thus it only rmeains to show that $M+M^\perp = H$. To this end, we fix $x \in H$. Since $M \subset H$ is a closed linear subspace (and therefore in particular convex) we can apply the previous theorem to find $z \in M$ such that dist$(x,M)=\|x-z\|$.
\\\\
We now claim that $x-z  \in M^\perp$, which gives that $x = z + (x-z)$ is the desired decomposition of $H$ into $M$ and $M^\perp$. Lets assume for contradiction that $(x-z) \notin M^\perp$. Then there exists $\alpha \in M$ with $(x-z, \alpha )>0$. For $t \in [-1,1]$ let $z_t = z + t \alpha$. Then we have $z_t \in M$ for all $t$ and
\begin{align*}
\|x-z_t\|^2 = \|x-z\|^2 + t^2\| \alpha\|^2 -2t(x-z, \alpha) < \|x-z\|^2 = \text{dist}(x,M)
\end{align*}
for $t >0$ small enough. But this condradicts the definition of dist$(x,M)$. 
\end{proof}
\begin{defn} An orthonormal system in $(H, ( \cdot, \cdot))$ is a family $(x_\alpha)_{\alpha \in A} \subset H$ for an arbitrary index-set $A$ with $(x_\alpha, x_\beta)= \delta_{\alpha, \beta}$. In the case when $A = \mathbb{N}$, we also call the orthonormal system an orthonormal sequence. 
\end{defn}
\begin{lem} Let $H$ be a Hilbert space, $(x_n)_{n \in \mathbb{N}}$ an orthonormal system (orthonormal sequence) and $(\alpha_n)_{n \in \mathbb{N}}$ a sequence in $\mathbb{K}$. Then we have
\begin{enumerate}
\item $\sum_{k=1}^\infty \alpha_k x_k$ converges if and only if $\sum_{k=1}^\infty | \alpha_k|^2$ converges. 
\item $\| \sum_{k=1}^n \alpha_k x_k\|^2 = \sum_{k=1}^n |\alpha_k|^2$.
\item If $\sum_{k=1}^\infty \alpha_k x_k$ converges, then the limit is independent of the order of the terms. 
\end{enumerate}
\end{lem}
\begin{lem} Let $(H, ( \cdot , \cdot))$ be a Hilbert space, $A$ an arbitrary set an $(x_\alpha)_{\alpha \in A}$ an orthonormal system in $H$. Then $\sum_{ \alpha \in A} ( x_\alpha ,x) x_\alpha$ converges for every $x \in H$. Moreover, the linear map $\phi : H \to H$ defined through $\phi(x) = \sum_{ \alpha \in A} (x_\alpha, x) x_\alpha$ is the continuous projection onto
\begin{align*}
M:= \overline{\text{span}\lbrace x_\alpha : \alpha \in A \rbrace } \text{ along its orthogonal complement } M^\perp
\end{align*}
In particular for $x \in M$, we find that $x = \sum_{ \alpha \in A} (x_\alpha, x) x_\alpha$. 
\end{lem}
\newpage
\begin{rem} In particular, if $M = \text{span} \lbrace x_\alpha :  \alpha \in A \rbrace$ is dense in $H$, i.e. if $\overline{M}=H$, then gives us the previous Lemma a representation for every vector $x \in H$. In this case, we say that $(x_\alpha)_{ \alpha \in A}$ is a Hilbert space basis. 
\end{rem}
\begin{defn} Let $H$ be a Hilbert space. A Hilbert space basis is an orthonormal system $(x_\alpha)_{ \alpha \in A}$ with \begin{align*}
\overline{\text{span} \lbrace x_\alpha : \alpha \in A \rbrace }=H
\end{align*}
\end{defn}
\begin{thm}[Characterizations of Hilbert space bases] Let $H$ be a Hilbert space, and $(x_\alpha)_{ \alpha \in A}$ an orthonormal system. Then the following statements are equivalent:
\begin{enumerate}
\item $(x_\alpha)_{ \alpha \in A}$ is a Hilbert space basis.
\item $x= \sum_{ \alpha \in A} (x_\alpha, x) x_\alpha$, for all $x \in H$.
\item $\|x\|^2 = \sum_{ \alpha} |(x_\alpha, x)|^2$ for all $x \in H$. 
\item $(x_\alpha, x)=0$ for all $\alpha \in A$ implies that $x=0$.
\item $(x_\alpha)_{\alpha \in A}$ is a maximal orthonormal system in the sense of inclusions. 
\end{enumerate}
\end{thm}
\begin{rem} Using the maximality property, i.e. point 5 in the theorem above, it follows easily from the Lemma of Zorn that every pre-Hilbert space has a Hilbert space basis. In particular, it follows that every separable Hilbert space admits a countable orthonormal basis. The observation that separable Hilbert spaces have countable orthonormal bases can be used to identify separable Hilbert spaces with the sequence space $\ell^2( \mathbb{K})$. 
\end{rem}
\begin{thm} Let $H$ be an infinite dimensional separable Hilbert space over $\mathbb{K}$. Then there exists a linear Isomorphism $\phi: H \to \ell^2( \mathbb{K})$ with 
\begin{align*}
( \phi(x), \phi(y))_{ \ell^2} =(x,y)_H
\end{align*}
for all $x,y \in H$. In particular the isomorphism is isometric. 
\end{thm}
\newpage
\section{Function Spaces}
\subsection{Continuous Functions on Compact Spaces}
\begin{lem} Let $f \in C_\mathbb{K}(K)$ where $K$ is compact. Then $f$ is bounded and its supremum and infimum are attained. 
\end{lem}
\noindent This allows us to define the sup/max norm of $f$ on the space $C_\mathbb{K}(K)$.
\begin{defn} For $f \in C_\mathbb{K}(K)$ let 
\begin{align*}
\|f\|:= \sup_{x \in K} | f(x)| = \max_{x \in K} | f(x)|
\end{align*}
\end{defn}
\noindent It is then simple to check that $\| \cdot \|$ defines a norm on $C_\mathbb{K}(K)$ (well-defined because of the above Lemma). Hence, the pair $(C_\mathbb{K}(K), \| \cdot \|)$ is a normed space. 
\begin{thm} $(C_\mathbb{K}(K), \| \cdot \|)$ is a Banach space. 
\end{thm}
\begin{proof}
Let $f_n$ be a Cauchy sequence in $C_\mathbb{K}(K)$. Then, trivially we have for any $x \in K$ that 
\begin{align*}
|f_n(x) - f_m(x)| \leq \|f_n -f_m \| \to 0 \text{ as } n,m \to \infty
\end{align*}
It follows that $(f_n(x))_{n \in \mathbb{N}}$ is a Cauchy Sequence in $\mathbb{K}$, but since $\mathbb{K}$ is complete we conclude that $f_n(x)$ converges. Let $f(x):= \lim f_n(x)$ denote its limit. We easily note that
\begin{align*}
|f(x)| \leq \limsup_{n \to \infty} |f_n(x)| \leq \limsup_{n \to \infty} \|f_n\|.
\end{align*}
Moreover, since for every $x \in K$, we have 
\begin{align*}
|f(x)-f_n(x)| = \lim_{m \to \infty} | f_m(x)-f_n(x)| \leq \limsup_{m \to \infty} \| f_n-f_m\|
\end{align*}
we conclude that
\begin{align*}
\sup_{x \in K} |f(x)-f_n(x)| \leq \limsup_{m \to \infty} \|f_m-f_n\| \to 0 \text{ as } n \to \infty.
\end{align*}
since this convergences in uniform over all $x \in K$, we conclude that $f$ is indeed continuous, thus our Cauchy sequence $f_n$ converges to $f \in C_\mathbb{K}(K)$ and our space is complete. 
\end{proof}
\newpage
\begin{defn} $\mathbb{A} \subset C_\mathbb{K}(K)$ is called a subalgebra if $\mathbb{A}$ is a linear subspace of $C_\mathbb{K}(K)$ and if, for every $f,g \in \mathbb{A}$, we also have $f \cdot g \in \mathbb{A}$. We say that the subalgebra $\mathbb{A}$ separates the points of $K$, if 
\begin{align*}
\forall x,y \in K \text{ with } x \neq y, \exists f \in \mathbb{A} : f(x) \neq f(y). 
\end{align*}
\end{defn}
\begin{rem} We already know that if $\mathbb{A}= C_\mathbb{K}(K)$ then indeed $C_\mathbb{K}(K)$ is a subalgebra separating the points of $K$. 
\end{rem}
\begin{thm}[Stone-Weierstrass Theorem, $\mathbb{K}= \mathbb{R}$] Let $\mathbb{A}$ be a subalgebra of $C_\mathbb{R}(K)$ separating the points of $K$. Then we have either $\overline{\mathbb{A}}= C_\mathbb{R}(K)$ or there exists a unique point $x_0 \in K$ such that $\overline{\mathbb{A}}= \lbrace f \in C_\mathbb{R}(K) : f(x_0)=0 \rbrace$. 
\end{thm}
\begin{rem} The Stone-Weierstrass Theorem allows us for example to approximate continuous functions on compact subsets of $\mathbb{R}^n$ through series of polynomials, see next example.
\end{rem}
\begin{exmp} Let $K \subset \mathbb{R}^n$ be compact and $\mathbb{A}$ be the set of all polynomials in the variables $x_1, \dots , x_n$, i.e. 
\begin{align*}
\mathbb{A}= \left\{ p(x) = \sum_{ \alpha : | \alpha| \leq n } b_\alpha x^\alpha, \ m \in \mathbb{N}, \ b_\alpha \in \mathbb{R} \right\} 
\end{align*}
It is then easy to check that $\mathbb{A}$ is a subalgebra of $C_\mathbb{K}(K)$, separating the points of $K$, moreover that $\lbrace f \in C_\mathbb{K}(K) : f(x_0)=0 \rbrace$ cannot hold true, because evidently we have $1 \in \mathbb{A}$ and if we were in the second case, then there would exists $x_0 \in K$ such that $0=1(x_0)=1$ which is a contradiction. Thus by Stone-Weierstrass we concludethat $\overline{\mathbb{A}} = C_\mathbb{K}(K)$. 
\end{exmp}
\begin{thm}[Stone-Weierstrass, $\mathbb{K}= \mathbb{C}$] Let $\mathbb{A}$ be a subalgebra of $C_\mathbb{C}(K)$, separating the points of $K$ and such that for all $f \in \mathbb{B}$ we have $\overline{f} \in \mathbb{A}$. Then we either have $\overline{\mathbb{A}}= C_\mathbb{C}(K)$ or 
\begin{align*}
\exists ! x_0 \in K \text{ with } \overline{\mathbb{A}}= \lbrace f \in C_\mathbb{C}(K) : f(x_0)=0 \rbrace. 
\end{align*}
\end{thm}
\subsection{Lebesgue Spaces}
This should already be known, we will only repeat the most relevant results without much explanation in this section. We work on some measure space $(\Omega, \Sigma, \mu)$ and consider the space 
\begin{align*}
\tilde{L}^p( \Omega, d \mu)= \lbrace f: \Omega \to \mathbb{R} : f \text{ measurable}, \int_\Omega |f|^p d \mu < \infty \rbrace 
\end{align*}
\begin{thm}[Hölder's inequality] Let $1 \leq p,q \leq \infty$ such that $\frac{1}{p} + \frac{1}{q}=1$. Let $( \Omega, \Sigma, \mu)$ be a measure space and $f \in L^p( \Omega), g \in L^q( \Omega)$, then $fg \in L^1(\Omega)$ and 
\begin{align*}
\left| \int_\Omega fg d \mu \right| \leq \int_\Omega |f| |g| d \mu \leq \|f\|_p \cdot \|g|_q
\end{align*}
\end{thm}
\newpage
\begin{thm}[Fischer-Riesz] Let $( \Omega, \Sigma, \mu)$ be a measure space and $1 \leq p \leq \infty$. Then $L^p( \Omega, \Sigma, \mu)$ is complete. 
\end{thm}
\begin{rem} For $p=2$ the Lebesgue space $L^2$ is even a Hilbert space with scalar product on $L^2( \Omega, \Sigma, \mu)$ given by 
\begin{align*}
(f,g)_{L^2} = \int \overline{f(x)} g(x) d\mu(x) 
\end{align*}
an exercise from class entails that this is on the only Lebesgue Space that is a Hilbert Space, i.e. only for the case $p=2$. 
\end{rem}
\begin{defn} Let $(X, \| \cdot \|)$ be a normed space over $\mathbb{K}$. We define the dual space $X^*$ of $X$ as the space of all continuous linear functionals on $X$, i.e. 
\begin{align*}
X^*:= \lbrace f: X \to \mathbb{K} \mid f \text{ is linear and continuous} \rbrace
\end{align*}
We will discuss this space in much more detail later on, for now we note that $X^*$ is always a Banach space w.r.t. the norm 
\begin{align*}
\|f\|_{X^*} := \sup_{x \in X: \|x\| = 1 } |f(x)| 
\end{align*}
\end{defn}
\begin{thm}[Duality of $L^p$ spaces] Let $1 \leq p < \infty, 1 < q \leq \infty$ be such that $\frac{1}{p} + \frac{1}{q}=1$ and let $( \Omega, \mathcal{A}, \mu)$ be a measure space. If $p=1$, we assume additionally that the measure space $( \Omega, \mathcal{A}, \mu)$ is $\sigma$-finite.  Then the map
\begin{align*}
\phi : \begin{cases} L^q(\Omega, \mathcal{A}, \mu) & \longrightarrow(L^p (\Omega, \mathcal{A}, \mu))^* \\
f & \longmapsto \phi_f \end{cases}
\end{align*}
with 
\begin{align*}
\phi_f(g) = \int_\Omega \overline{f(x)}g(x) d \mu(x)
\end{align*}
for all $g \in L^p(\Omega, \mathcal{A}, \mu)$ is an anti-linear isometric isomorphism. In other words, we have
\begin{align*}
L^p(\Omega, \mathcal{A}, \mu)^* \cong L^q(\Omega, \mathcal{A}, \mu)
\end{align*}
\end{thm}
\begin{rem} The theorem determines the dual space of $L^p(\Omega, \mathcal{A}, \mu)$, it shows that every continuous linear functional $L$ on $L^p(\Omega, \mathcal{A}, \mu)$ for $1 \leq p < \infty$ has the form 
\begin{align*}
L(f)= \int_\Omega \overline{g(x)} f(x) d \mu(x)
\end{align*}
for a $g \in L^q(\Omega, \mathcal{A}, \mu)$ with $\frac{1}{p} + \frac{1}{q}=1$. 
\end{rem}
\newpage
Our last goal in this section is to show that smooth functions are dense in $L^p$, for all $1 \leq p < \infty$. For simplicity we work on $\mathbb{R}^n$, then the next theorem shows how to approximate $L^p$-functions with sequences of $C^\infty$-functions.
\begin{thm} Let $j \in L^1 ( \mathbb{R}^n)$ with $\int j d \lambda_n(x) =1$. For $\epsilon >0$ we let $j_\epsilon(x) = \epsilon^{-1} j(x/ \epsilon)$ so that $\int j_\epsilon d\lambda_n =1$ for all $\epsilon >0$. Let now $f \in L^p( \mathbb{R}^n)$ for a $1 \leq p < \infty$ and set $f_ \epsilon = f * j_\epsilon$. Then $f_\epsilon \in L^p( \mathbb{R}^n)$ with $\|f_\epsilon \|_p \leq \|j\|_1 \|f\|_p$ and $f_\epsilon\to f $ in $L^p( \mathbb{R}^n)$ as $\epsilon \to 0$. If $j \in C_c^\infty ( \mathbb{R}^n)$, then we have $f_\epsilon \in C^\infty ( \mathbb{R}^n)$ and $D^\alpha f_\epsilon = (D^\alpha j_\epsilon) * f$ for all multi-indices $\alpha= ( \alpha_1, \dots, \alpha_n) \in \mathbb{N}^n$. 
\end{thm}
\begin{lem} Let $\Omega \subset \mathbb{R}^n$ open, $K \subset \Omega$ compact. Then there exists $J_K \in C_c^\infty( \Omega)$ with $0 \leq J_K(x) \leq 1$ for all $x \in \Omega, \ J_K(x)=1$ for all $x \in K$. Therefore, there exists a sequence $(g_j)_{j \in \mathbb{N}}$ in $C_C^\infty( \Omega)$ with $0 \leq g_j(x) \leq 1$ for all $j \in \mathbb{N}$ and $\lim g_j(x) = 1$ for all $x \in \Omega$. Hence if $(f_j)_{j \in \mathbb{N}}$ is a sequence in $C^\infty( \Omega)$ with $f_j \to f$ in $L^p( \Omega)$ for $1 \leq p < \infty$, then $g_jf_j \in C_c^\infty( \Omega)$ and $g_jf_j \to f$ in $L^p( \Omega)$. The Theorem above implies therefore that $C_c^\infty(\Omega)$ is dense in $L^p( \Omega)$ for all $1 \leq p < \infty$.
\end{lem}
As an application of the density of $C_c^\infty( \mathbb{R}^n) $ in $L^p( \mathbb{R}^n)$ we have the following theorem.
\begin{thm} For every measurable $\Omega \subset \mathbb{R}^n$ and $1 \leq p < \infty$, the Banach space $L^p( \Omega)$ is seperable. 
\end{thm}
\subsection{Sobolev Spaces}
Functions in $L^p$-spaces are characterized by their integrability properties. In analysis, however, it is often important to consider derivatives. The notion of derivative does not combine very well with $L^p$ spaces, functions in $L^p$ spaces are typically not differentiable. For this reason, we are going to introduce a weaker notion of derivative, Sobolev spaces consists then of $L^p$-functions, whose weak derivative is again an $L^p$-function. 
\\\\
We work with $\Omega \subset \mathbb{R}^n$ open and let $ \Sigma \subset 2^\Omega$ be the Borel $\sigma$-algebra over $\Omega$, moreover let $dx$ denotes Lebesgue mass. We define the normed space 
\begin{align*}
X = \lbrace f \in C^\infty( \Omega): \|f\|_X < \infty \rbrace, \text{ with } \|f\|_X = \sum_{ | \alpha| \leq m } \| D^\alpha f \|_{L^p(\Omega)}
\end{align*}
where the sum above runs over all $\alpha = ( \alpha_1, \dots , \alpha_n) \in \mathbb{N}^n$ with $|\alpha| = \sum_{i=1}^n \alpha_i$ and $D^\alpha f = \partial_1^{\alpha_1} \cdots \partial_n^{\alpha_n} f$. It is easy to check that $(X, \| \cdot \|_X)$ is indeed a normed space but that it is not complete. We define $\tilde{X}$ as the completion of this normed space and we want to find a good cahracterization of $\tilde{X}$.
\\\\
To this end, let $[(f_j)_{j \in \mathbb{N}}] \in \tilde{X}$, i.e. let $(f_j)_{j \in \mathbb{N}}$ be a Cauchy sequence in $X$.  Then by the definition of the norm on $X$, $(D^\alpha f_j)_{j \in \mathbb{N}}$ defines a Cauchy sequence on $L^p( \Omega)$ because we have for all $\alpha \in \mathbb{N}^n$ with $| \alpha| \leq m$
\begin{align*}
\| D^\alpha f_j - D^\alpha f_l \|_{L^p} \leq \| D^\alpha f_j - D^\alpha f_l\|_X \to 0 \text{ as } j,l \to \infty. 
\end{align*}
Since $L^p( \Omega)$ is a complete space, there exiss $f^{(\alpha)} \in L^p( \Omega)$ such that $D^\alpha f_j \to f^{(\alpha)}$ in $L^p$ as $j \to \infty$. For any $\xi \in C_c^\infty( \Omega)$ we obtain by the above convergence, integration by parts and the fact that $\xi$ is compactly supported that 
\begin{align*}
\int_\Omega D^\alpha \xi f^{(0)} dx &= \lim_{j \to \infty} \int_\Omega D^\alpha \xi f_j dx  \\ &= (-1)^ {| \alpha|} \lim_{j \to \infty}  \int_\Omega \xi D^\alpha f_j dx = (-1)^{| \alpha|} \int_\Omega \xi f^{( \alpha)} dx
\end{align*}
This identity gives a relationship between $f^{( \alpha)}$ and $f^{(0)}=f$, for all $\alpha \in \mathbb{N}^n$ with $| \alpha | \leq m$. \\
\\
Motivated by this remark, we define the Sobolev space of order $m \in \mathbb{N}$ with exponent $1 \leq p \leq \infty$ as 
\begin{align*}
H^{m,p}( \Omega) = \lbrace f \in L^p( \Omega) : \forall \alpha \in \mathbb{N}^n \text{ with } 1 \leq | \alpha | \leq m, \exists f^{( \alpha)} \in L^p( \Omega) \text{ such that } \\ \int_\Omega D^\alpha \xi f dx = (-1)^{| \alpha|} \int_\Omega \xi f^{( \alpha)} dx \text{ for all } \xi \in C_c^\infty ( \Omega) \rbrace
\end{align*}
We equip $H^{m,p}( \Omega)$ with the norm
\begin{align*}
\|f\|_{H^{m,p}} := \sum_{| \alpha| \leq m } \|f^{( \alpha)} \|_{L^p}.
\end{align*}
It is then easy to check that $(H^{m,p}( \Omega), \| \cdot \|_{H^{m,p}})$ is a normed space. For $f \in H^{m,p}( \Omega)$ we call the functions $f^{( \alpha)}, 1 \leq | \alpha | \leq m$, the weak derivatives of $f$. We often use the notation $f^{( \alpha)} = \partial^\alpha f$. The following Lemma shows that the weak derivatives of a function are unique.
\begin{lem} Let $\Omega \subset \mathbb{R}^n$ open and $f \in L^p( \Omega, dx)$, $1 \leq p \leq \infty$, with 
\begin{align*}
\int_\Omega \xi f dx =0, \text{ for all } \xi \in C_c^\infty( \Omega)
\end{align*}
then we have $f=0$. 
\end{lem}
\begin{rem} If $f \in L^p( \Omega) \cap C^m( \Omega)$, then the classical derivates $D^\alpha f$ coincide with the weak derivatives, this can be shown by using integration by parts. The advantage of the weak derivatives is of course the fact that they exist for a much larger class of functions.
\end{rem}
\noindent Next we show that the completeness of the $L^p$ spaces implies also the completeness of the Sobolev spaces. 
\newpage 
\begin{thm} For all $m \in \mathbb{N}, \ 1 \leq p \leq \infty, \ H^{m,p}( \Omega)$ is a Banach space. 
\end{thm} 
\begin{proof}
Let $(f_j)_{j \in \mathbb{N}}$ be a Cauchy sequence in $H^{m,p}( \Omega)$, then clearly $f_j$ is also a Cauchy sequence in $L^p( \Omega)$, hence there exists $f \in L^p( \Omega)$ with $f_j \to j$ in $L^p(\Omega)$. Moreover, for every $\alpha \in \mathbb{N}^n$ with $| \alpha| \leq m$, we also have that $\partial^\alpha f$ is a Cauchy sequence in $L^p(\Omega)$. Hence there exists $g_\alpha \in L^p( \Omega)$ with $\partial^\alpha f_j \to g_\alpha$ in $L^p(\Omega)$. We claim that that in fact $f \in H^{m,p}(\Omega)$ and that $\partial^\alpha f = g_\alpha$. In fact, we observe that for arbitray $\xi \in C_c^\infty(\Omega)$ we have
\begin{align*}
\int_{\Omega} D^\alpha \xi f dx = \lim_{j \to \infty} \int_{\Omega} D^\alpha \xi f_j dx = (-1)^{| \alpha|} \lim_{j \to \infty} \int_\Omega \xi \partial^\alpha f_j dx = (-1)^{|\alpha|} \int_\Omega \xi g_\alpha dx
\end{align*}
\end{proof}
\begin{thm}[Approximation Theorem] Let $f \in H^{m,p}(\Omega)$, $1 \leq p < \infty$, $m \in \mathbb{N}$. Then there exists a sequence in $H^{m,p}(\Omega) \cap C^\infty( \Omega)$ with $\| f_j-f\|_{H^{m,p}}$ as $j \to \infty$. Moreover, if $\Omega = \mathbb{R}^n$, then we can choose $f_k \in C_c^\infty( \mathbb{R}^n)$ with $\|f_k-f\|_{H^{m,p}} \to 0$ as $k \to \infty$.
\end{thm}
\begin{rem} The theorem shows that, for all $1 \leq p < \infty$, $H^{m,p}(\Omega)$ is the completion of $C^\infty(\Omega)$ with respect to the norm of $H^{m,p}$. 
\end{rem}
\section{Compactness}
\subsection{Compact Sets in Metric Spaces}
\begin{defn} Let $(X,d)$ be a metric space. A subset $A \subset X$ is called pre-compact if, for all $\epsilon >0$, $A$ has a finite covering with $\epsilon$-balls, i.e. if there are $x_1, \dots , x_n \in X$ with $\cup_{i=1}^n B_\epsilon(x_i) \supset A$. 
\end{defn}
\begin{rem} We have the following
\begin{enumerate}
\item Subsets of pre-compact sets are pre-compact.
\item $A \subset X$ pre-compact implies that $A$ is bounded.
\item $A \subset X$ pre-compact implies that $\overline{A}$ is closed and pre-compact. 
\end{enumerate}
\end{rem}
\begin{thm} Let $(X,d)$ be a metric space. For $A \subset X$, the following statements are equivalent:
\begin{enumerate}
\item $A$ is compact.
\item $A$ is sequentially compact, i.e. every sequence in $A$ has a convergent subsequence.
\item $(A,d)$ is complete and $A$ is pre-compact. 
\end{enumerate}
\end{thm}
\newpage
\begin{rem} \ \begin{enumerate}
\item We see that what's missing for a pre-compact set to be compact is completeness. 
\item $A \subset X$ compact implies that $A$ is closed (since $A$ complete implies that $A$ is closed)
\item If $X$ is complete, then $A \subset X$ is pre-compact if and only if $\overline{A}$ is compact (since closed subsets of complete spaces are complete). 
\end{enumerate}
\end{rem}
\noindent On finite dimensional vector spaces (for example normed spaces with finite dimensional $V$ vector space) all norms are equivalent. 
\begin{lem} Let $X$ be a finite dimensional vector space over $\mathbb{K}$. Then all norms on $X$ are equivalent. In other words, if $\| \cdot \|_1,  \| \cdot \|_2$ are two norms on $X$, then there exists $C>0$ such that 
\begin{align*}
\frac{1}{C} \|x\|_2 \leq \|x\|_1 \leq C \|x\|_2, \text{ for all } x \in X. 
\end{align*}
\end{lem}
\begin{cor} Every finite dimensional subspace of a normed space is complete and therefore closed. 
\end{cor}
\begin{thm}[Heine-Borel] Let $(X, \| \cdot\|)$ be a finite dimensional vector space. Then $A \subset X$ is compact if and only if $A$ is bounded and closed.  
\end{thm}
\begin{proof}
"$\implies$" Readily follows from the Theorem above, because $A$ is compact if and only if $(A,d)$ is complete and $A$ is pre-compact, which implies that $A$ is bounded and closed. 
\\\\
"$\Longleftarrow$" Assume that $A \subset X$ is bounded and closed. Since $X \subset X$ is a finite dimensional subspace, $X$ is complete and therefore closed. Thus it is enough to show that $A \subset X$ is pre-compact and we know that this is the case if and only if $\overline{A}=A$ is complete. 
\\\\
Let $\lbrace e_1, \dots , e_n \rbrace$ be a basis of $X$. Then we define the norm \begin{align*}
\|x\|_\infty = \max_{j=1, \dots , n} | x_j|, \text{ for } x = \sum_{k=1}^n x_j e_j \in X
\end{align*}
on $X$. Since all norms are equivalent, it is enough to prove the claim w.r.t. to this norm $\| \cdot \|_\infty$. To this end, it is enough to show that for $R>0, \epsilon >0$ there exists $n(R, \epsilon) \in \mathbb{N}$ and $x^{(1)}, \dots , x^{(n(R, \epsilon))} \in X$ such that \begin{align*}
B_R(0) \subset \bigcup_{j=1}^{n(R, \epsilon)} B_\epsilon (x^{(j)}) 
\end{align*}
\end{proof}
\begin{prop} \label{compactunitball} Let $X$ be a normed space. Then $\overline{B_1(0)}$ is compact if and only if $\dim X < \infty$
\end{prop}

\subsection{Compact Subsets of $C(K)$}
For finite dimensional normed spaces we found a simple characterization of compactness, namely if $\dim X < \infty$ then $A \subset X$ is compact if and only if $A$ is closed and bounded. We try to find a similar simple characterization of compact subsets for the case of infinite dimensional spaces. We denote $C(K)=C_\mathbb{K}(K)$ where $\mathbb{K}= \mathbb{R}$ or $\mathbb{C}$. 
\begin{defn} $S \subset C(K)$ is called equicontinuous at $x \in K$ if 
\begin{align*}
\forall \epsilon >0, \exists U_x \text{ open neighbourhood of $x$ in $K$} : |f(x)-f(y)| \leq \epsilon, \forall y \in U_x, \forall f \in S. 
\end{align*}
$S \subset C(K)$ is called equicontinuous, if $S$ is equicontinuous at $x$, for all $x \in K$. 
\end{defn}
\begin{rem} The open neighbourhood of $x$ $U_x$ may solely depend on $\epsilon$ and the point $x$, it mustn't depend on $f \in S$. 
\end{rem}
\begin{thm}[Arzelá-Ascoli] A subset $S \subset C(K)$ is pre-compact if and only if $S$ is bounded and equicontinuous. 
\end{thm}
\begin{rem} We know that $S \subset C(K)$ is pre-compact if and only if $\overline{S}$ is compact. Hence from Arzelá-Ascoli we obtain that $S \subset C(K)$ is compact if and only $S$ is closed, bounded and equicontinuous. In particular, in comparison with Heine-Borel Theorem we require to satisfy 1 additional conditional, namely the condition of equicontinuity. 
\end{rem}
\begin{proof}
"$\implies$" Let $\epsilon >0$. Since $S$ is pre-compact, it has a finite covering with $\epsilon$-balls. In other words, there exists $f_1, \dots , f_n \in S$ with $S \subset \cup_{j=1}^n B_\epsilon(f_j)$. Hence, for all $f \in S$, there exists $j \in \lbrace 1, \dots , n \rbrace$ such that $f \in B_\epsilon ( f_j)$ and therefore for all $f \in S$ 
\begin{align*}
\| f\|\ \leq \| f_j\| + \epsilon
\end{align*}
This proves that 
\begin{align*}
\sup_{j \in S} \|f \| \leq \max_{j=1, \dots , n} \|f_j\| + \epsilon
\end{align*}
and also that $S$ is bounded. Now, fix $x \in K$. Since the $f_i$ are continuous, we have for $i=1, \dots , n$ there exists an open neighbourhood $U_i$ of $x$ in $K$ such that 
\begin{align*}
|f_i(x)-f_i(y)| \leq \epsilon, \ \forall y \in U_i
\end{align*}
Let now $U:= \cap_{i=1}^n U_i$. Then $U$ is an open neighbourhood of $x \in K$ for all $f \in S$ and for all $y \in U$ we have 
\begin{align*}
|f(x)-f(y)| & \leq |f(x)-f_i(x)| + |f_i(x)-f_i(y)| + |f_i(y)-f(y)|  \\
& \leq 2 \|f-f_i\| + |f_i(x)-f_i(y)| \leq 3 \epsilon
\end{align*}
for an appropriate choice of $i \in \lbrace 1, \dots , n\rbrace$. This shows the equicontinuity of $S$. 
\newpage
For "$\Longleftarrow$", since $S$ is equicontinuous we can for $\epsilon >0$ and $x \in K$ always find $U_x$ open neighbourhood of $x$ such that 
\begin{align*}
|f(x)-f(y)| \leq \epsilon \ \forall f \in S, \forall y \in U_x.
\end{align*}
Then $\lbrace U_x\rbrace_{x \in K}$ is an open covering of $K$. I.e. there exists $x_1, \dots , x_n \in K$ with $K = \cup_{j=1}^n U_{x_j}$. Since moreover $S$ is bounded, we have that 
\begin{align*}
R= \lbrace (f(x_1), \dots , f(x_n)) : f \in S \rbrace
\end{align*}
is a bounded subset of $\mathbb{C}^n$, equipped with the max-norm. Therefore, $R$ is pre-compact in $\mathbb{C}^n$, i.e. there exists $f_1, \dots , f_m \in S$, with 
\begin{align*}
R \subset \bigcup_{i=1}^m B_\epsilon ( (f_i(x_1), \dots , f_i(x_n))
\end{align*}
we can than show that 
\begin{align*}
S \subset \bigcup_{i=1}^m B_{ 3 \epsilon} (f_i).
\end{align*}
which shows that $S$ is pre-compact. 
\end{proof}
\subsection{Compact Subsets of $L^p$-Spaces}
\begin{thm}[Riesz's Theorem] Let $1 \leq p < \infty$, $A \subset L^p( \mathbb{R}^n)$. Then $A$ is pre-compact if and only if the following conditions are satisfied:
\begin{enumerate}
\item $A$ is bounded 
\item $\displaystyle \sup_{f \in A} \|f( \cdot + h) -f \|_p \to 0$ as $h \to 0$
\item $\displaystyle \sup_{f \in A} \|f\|_{L^p( \mathbb{R}^N \setminus B_r(0)) } \to 0$ as $R \to \infty$.
\end{enumerate}
\end{thm}
\begin{rem} The theorem holds true for subsets of $L^p( \mathbb{R}^n)$. There are also extensions to subsets of $L^p( \Omega)$, for measurable $\Omega \subset \mathbb{R}^n$. 
\end{rem}
\newpage
\section{Linear Operators \& Functionals on Normed Spaces}
\subsection{Continuous Operators}
\begin{defn} Let $X,Y$ be two normed spaces. A (linear) operator $T: X \to Y$ is a linear map. A continuous operator $T:X \to Y$ is a continuous linear map. An operator $T: X \to Y$ is called bounded, if there exists a constant $C>0$ with 
\begin{align*}
\|Tx\|_Y \leq C \|x \|_X, \text{ for all } x \in X
\end{align*}
\end{defn}
\begin{prop}[Characterization of continuous opetaros] Let $X,Y$ be normed spaces and $T: X \to Y$ a linear operator. The following statements are equivalent:
\begin{enumerate}
\item $T$ is continuous.
\item $T$ is bounded.
\item $T$ is continuous at $x=0$. 
\end{enumerate}
\end{prop}
\begin{defn} Let $X,Y$ be normed spaces and $\mathbb{K}$ be $\mathbb{R}$ or $\mathbb{C}$. Then we denote by $\mathcal{L}(X,Y)$ the space of all continuous linear operators from $X$ to $Y$. $\mathcal{L}(X,Y)$ has the structure of a vector space over $\mathbb{K}$ when we consider the canonical operators for additions and multiplication with scalars. Moreover, for $T \in \mathcal{L}(X,Y)$ we define
\begin{align*}
\|T\| := \sup_{x \neq 0} \frac{\|Tx\|}{\|x\|}= \sup_{\|x \| \leq 1} \| Tx\| = \sup_{\|x\|=1} \|Tx\|
\end{align*}
in particular $(\mathcal{L}(X,Y), \| \cdot \|)$ is a normed space. We also denote $\mathcal{L}(X)=\mathcal{L}(X,X)$.
\end{defn}
\begin{prop} Let $X$ be a normed space and $Y$ be a Banach space. Then $\mathcal{L}(X,Y)$ is a Banach space. 
\end{prop}
\begin{proof}
Let $(T_l)_{l \in \mathbb{N}}$ be a cauchy sequence on $\mathcal{L}(X,Y)$ and $x \in X$. Then we always have the inequality (!) 
\begin{align*}
\|T_lx - T_kx\| \leq \|T_l-T_k\| \|x\| \to 0 \text{ as } l,k \to \infty.
\end{align*}
Thus $(T_lx)_{l \in \mathbb{N}}$ is a Cauchy sequence in $Y$. Since $Y$ is by assumption complete, there exists \begin{align*}
Tx:= \lim_{l \to \infty} T_lx
\end{align*}
The map $T: X \to Y$ given as above is clearly linear. Moreover, $T \in \mathcal{L}(X,Y)$ since 
\begin{align*}
\|Tx\| = \lim_{l \to \infty} \|T_lx \|  \leq \limsup_{l \to \infty} \|T_l\| \|x\|  \leq C \|x\|,
\end{align*}
since $(T_l)_{l \in \mathbb{N}}$ is a Cauchy sequence and therefore bounded. We finally get 
\begin{align*}
\|Tx-T_lx\| = \lim_{k \to \infty} \|T_kx -T_lx\| \leq \limsup_{k \to \infty} \|T_k-T_l\|\|x\| \to 0 \text{ as } l \to \infty
\end{align*}
because $(T_k)_{k \in \mathbb{N}}$ is a Cauchy sequence, this completes the proof. 
\end{proof}
\begin{defn} Let $X$ be a normed space over the field $\mathbb{K}$. We define the dual space $X^*$ of $X$ through $X^*:= \mathcal{L}(X, \mathbb{K})$. That is elements of $X^*$ are continuous  linear functionals (i.e. continuous linear operators on $X$ with values on $\mathbb{K}$). Since $\mathbb{K}$ is always complete, it follows that $X^*$ is always a banach space, independentely of the completeness of $X$. 
\end{defn}
\subsection{The Hahn-Banach Theorem and its Applications}
Let us recall again the statement of the Lemma of Zorn. A partial order on a set $P$ is a relation $\preceq$ on $P$ with the following properties: 
\begin{enumerate}
\item $a \preceq a$ (Reflexivity)
\item $a \preceq b$ and $b \preceq a$ implies that $a=b$ (Antisymmetry)
\item If $a \preceq b$ and $b \preceq c$ then $a \preceq c$ (Transitivity)
\end{enumerate}
A subset $M \subset P$ of a partially ordered set $P$ is called totally ordered (or a chain) if for all $a,b \in M$ with $a \neq b$ we either have $a \preceq b$ or $b \preceq a$. Moreover, an element $b \in P$ is called an upper bound for $M \subset P$ if $a \preceq b$ for all $a \in M$. Naturally, we call $b \in P$ a maximal element in $P$, if there is no other upper bound $a \in P$ with $a \neq b$ and $b \leq a$ for said upper bound. 
\begin{lem}[Lemma of Zorn] Let $P$ be a partially ordered set, so that every totally ordered subset of $P$ has an upper bound. Then $P$ contains at least one maximal element. 
\end{lem}
\begin{defn} Let $X$ be a vector space over $\mathbb{R}$. A map $p: X \to \mathbb{R}$ is called a sublinear functional if 
\begin{enumerate}
\item $p( \lambda x) = \lambda p(x)$ for all $x \in X$ and $ \lambda \geq 0$. 
\item $p(x+y) \leq p(x) + p(y)$ for all $x,y \in X$. 
\end{enumerate}
\end{defn}
\begin{lem} Let $X$ be a vector space over $\mathbb{R}$. $M \subset X$ a linear subspace, $p:X \to \mathbb{R}$ a sublinear functional, $f: M \to \mathbb{R}$ a linear functional, $x_0 \in X \setminus M$. Assume $f(x) \leq p(x)$ for all $x \in M$. Then there exists $F: \widetilde{M}:= M + \mathbb{R}x_0 \to \mathbb{R}$ linear,  with $F(x) \leq p(x)$ for all $x \in \widetilde{M}$ and $F_{ \mid M} = f.$
\end{lem}
\newpage
\begin{thm}[Hahn-Banach, vector spaces over $\mathbb{R}$] Let $X$ be a vector space over $\mathbb{R}$, $M \subset X$ a linear subspace and $f: M \to \mathbb{R}$ linear. Let $p: X \to \mathbb{R}$ be a sublinear functional with $f(x) \leq p(x)$ for all $x \in M$. Then there exists $F: X \to \mathbb{R}$ linear with $F(x) \leq p(x)$ for all $x \in X$ and $F_{ \mid M} = f$. 
\end{thm}
\begin{proof}
We consider the set
\begin{align*}
\mathcal{F}:= \lbrace ( \widetilde{M}, \tilde{f}) : \widetilde{M} \subset X \text{ linear with } \widetilde{M} \supset M, \tilde{f} \text{ a linear function on } \widetilde{M}, \\ \text{with } \tilde{f}_{ \mid M} = f \text{ and } \tilde{f}(x) \leq p(x) \text{ for all } x \in \widetilde{M} \rbrace 
\end{align*}
We have $\mathcal{F} \neq \emptyset$ since by assumption $(M,f) \in \mathcal{F}$. On $\mathcal{F}$ we define a partial order by setting $(\widetilde{M}, \tilde{f}) \preceq ( \widetilde{N}, \tilde{g})$ if $\widetilde{M} \subset \widetilde{N}$ and $\tilde{g}_{ \mid \widetilde{M}} = \tilde{f}$. We claim that every totally ordered subset of $\mathcal{F}$ has an upper bound. \\
\\
To this end let $\mathcal{G}= \lbrace ( M_i, g_i): i \in I \rbrace$ be a totally ordered subset of $\mathcal{F}$. Since $\mathcal{G}$ is totally ordered we can define the linear space
\begin{align*}
\widetilde{M} = \bigcup_{i \in I} M_i
\end{align*}
and $\tilde{g}: \widetilde{M} \to \mathbb{R}$, so that for $x \in M_i, \tilde{g}(x)= g_i(x)$. We claim that $\tilde{g}$ is linear. Indeed, since $\mathcal{G}$ is totally ordered we have for any given $x,y \in \widetilde{M}$ and $\lambda \in \mathbb{R}$ there exists $i \in I$ (since $\mathcal{G}$ is totally ordered) such that $x,y \lambda y \in M_i$, hence we have 
\begin{align*}
\tilde{g}(x + \lambda y ) = \tilde{g}(x) + \lambda \tilde{g}(y)
\end{align*}
since the $g_i$ are linear. Moreover we clearly have $\tilde{g}_{ \mid M} = f$ and $\tilde{g}(x) \leq p(x)$ for all $x \in \widetilde{M}$. \\
\\
Thus $(\widetilde{M}, \tilde{g}) \in \mathcal{F}$ is an upper bound for $\mathcal{G}$. The Lemma of Zorn implies that there exists a maximal element $(N,F) \in \mathcal{F}$. We want to establish that $N=X$. Assume that $N \neq X$ then $X \setminus N \neq \emptyset$ and we can choose $x_0 \in X \setminus N$, if we apply now the previous Lemma to said $x_0$ then there exists $G$ with $(N+ \mathbb{R}x_0, G) \in \mathcal{F}, G_{ \mid N} = F$. But we clearly have $(N,F) \preceq (N + \mathbb{R}x_0, G)$ and $(N,F) \neq (N + \mathbb{R}x_0, G)$. This is a contradiction to the maximilty of $(N,F)$ and thus $N=X$.  
\end{proof}
\noindent We want to extend the theorem to vector spaces over $\mathbb{C}$, we thus need to intrtoduce the notion of a seminorm. 
\begin{defn} Let $X$ be a vector space over $\mathbb{K}$ and $q: X \to \mathbb{R}$ a map with the properties $q( \alpha x) = | \alpha | q(x)$ for all $\alpha \in \mathbb{K}, x \in X$ and with $q(x+y) \leq q(x) + q(y)$ for all $x,y \in X$. Then we call $q$ a seminorm. 
\end{defn}
\newpage
\begin{thm}[Hahn-Banach, general version, for real and complex vector spaces] Let $X$ be a vector space over $\mathbb{K}= \mathbb{R}$ or $\mathbb{K}= \mathbb{C}$, $q :X \to \mathbb{R}$ a seminorm, $M \subset X$ a linear subspace and $f: M \to \mathbb{K}$ a linear functional with $|f(x)| \leq q(x)$ for all $x \in M$. Then there exists a linear functional $F: X \to \mathbb{K}$ with $F_{ \mid M} = f$ and $|F(x)| \leq q(x)$ for all $x \in X$. 
\end{thm}
\begin{proof}
If $\mathbb{K}= \mathbb{R}$ we can apply Hahn-Banach Theorem (real version) to find $F: X \to \mathbb{R}$ linear with $F_{ \mid M} =f$ and with $F(x) \leq q(x)$ for all $x \in X$. Then we have
\begin{align*}
-F(x) = F(-x) \leq q(-x)=q(x) \implies F(x) \geq -q(x) \text{, thus } |F(x)| \leq q(x).
\end{align*}
So, let us now consider the case when $\mathbb{K}= \mathbb{C}$. We can consider $X$ and $M$ as vector spaces over $\mathbb{R}$. Notice that Re $f: M \to \mathbb{R}$ is $\mathbb{R}$-linear with $|\text{Re } f(x) | \leq |f(x)| \leq q(x)$. We apply Hahn-Banach for vector spaces over $\mathbb{R}$ and find $G: X \to \mathbb{R}$, $\mathbb{R}$-linear with $G_{ \mid M}$ = Re $f$ and with $|G(x)| \leq q(x)$ for all $x \in X$. 
\\\\
Let us now define $F(x) = G(x)-iG(ix)$. Then $F$ is clearly $\mathbb{R}$ liniear, however since we have \begin{align*}
F(ix)&=G(ix)-iG(i^2x) = G(ix)-iG((-1)x)=G(ix)+iG(x) \\
& = iG(x) +G(ix) = i(G(x)-iG(ix))=iF(x)
\end{align*}
we can also conclude that $F$ is $\mathbb{C}$-linear. Moreover we have for $x \in M$ 
\begin{align*}
\text{Re } F(x)&=G(x)=\text{Re } f(x) \\
\text{Im } F(x)& = -G(ix)= - \text{Re } f(ix) = - \text{Re } if(x) = \text{Im } f(x)
\end{align*}
This shows that $F_{ \mid M } = f$. We still need to verify that $|F(x)| \leq q(x)$ for all $x \in X$. Let $x \in X$ be arbitraryand we write $F(x)=re^{i \theta}$. Then $e^{-i \theta} F(x)$ is real and therefore
\begin{align*}
|F(x)| = |e^{-i \theta} F(x)| = |F(e^{-i \theta} x) | \overset{\text{real}}= | G(e^{- i \theta} x)| \leq q(e^{- i \theta x}) = q(x). 
\end{align*}
\end{proof}
\noindent The following corollaries are immediate consequences of the Hahn-Banach theorem. Of course, Hahn-Banach's Theorem is used to construct linear functionals. 
\newpage
\begin{cor} Let $(X, \| \cdot \|)$ be a normed space over $\mathbb{K}$, $M \subset X$, a linear subspace and $f \in M^*$. Then there exists $F \in X^*$ with $F_{ \mid M} = f$ and $\|F\|_{X^*} = \|f\|_{M^*}$. 
\end{cor}
\begin{proof}
We define $q(x)= \|f\|_{M^*} \|x\|$. Then $q$ is clearly a seminorm and we have $|f(x)| \leq q(x)$ for all $x \in M$. By Hahn-Banach, there exists a linear functional $F: X \to \mathbb{K}$ with $F_{ \mid M} = f$ and $|F(x)| \leq q(x) = \|f\|_{M^*} \|x\|$ for all $x \in X$. This readily implies that $F \in X^*$. Moreover it shows that $\|F\|_{X^*} \leq \|f\|_{M^*}$. On the other hand we have 
\begin{align*}
\|f\|_{M^*} = \sup_{x \in M, \|x\| \leq 1} |f(x)| = \sup_{x \in M, \|x \| \leq 1} | F(x)| \leq \sup_{x \in X, \|x \| \leq 1 } |F(x)| = \|F\|_{X^*}
\end{align*}
Hence we established $\|F\|_{X^*} = \|f\|_{M^*}.$
\end{proof}
\begin{cor} Let $(X, \| \cdot \|)$ be a normed space over $\mathbb{K}$, $y \in X \setminus \lbrace 0 \rbrace$. Then there exists $f \in X^*$ with $\|f\|=1$ and $f(y)= \|y\|$. 
\end{cor}
\begin{proof} We define $g: \mathbb{K} \cdot y =: M \to  \mathbb{K}$ through $g(ty)= t \|y\|$. Then clearly $g \in M^*$ with $\|g\|_{M^*}=1$ and $g(y)= \|y\|$. From the above Corollarly, there exists $f \in X^*$ with $f_{ \mid M} = g$ and hence in particular $f(y)=\|y\|$ and moreover $\|f\|_{X^*}=\|g\|_{M^*}=1$.
\end{proof} 
\begin{cor} Let $(X, \| \cdot \|)$ be a normed space over $\mathbb{K}, Z \subset X$ a linear subspace and $y \in X \setminus Z$. Let $d= \text{dist}(y,Z)= \inf_{z \in Z} \| z-y\| >0$. Then there exists $F \in X^*$ with $\|F\|=1$, $F_{ \mid Z} =0$ and $F(y)=d$. 
\end{cor}
\begin{proof}
Let $M= Z + \mathbb{K}y$ and define $f: M \to \mathbb{K}$ through $f(z + \alpha y) = \alpha d$, for all $\alpha \in \mathbb{K}$. Then $f$ is clearly linear. We claim that $\|f\|=1$. In fact, since for $\alpha \in \mathbb{K}$ and $z \in Z$ we have that $-z/ \alpha \in Z$ we can conclude that
\begin{align*}
|f(z+ \alpha y)|= | \alpha| d \leq | \alpha| \cdot \| y-(-z/ \alpha)\| = \| \alpha y + z \|
\end{align*}
In other words, we have $\|f\|_{M^*} \leq 1$. On the other hand, let $(z_n) \in Z$ be a sequence with $\| y -z_n\| \to d$. Then 
\begin{align*}
d = f(y-z_n) \leq \| f\|_{M^*} \| y -z_n \|
\end{align*}
As $n \to \infty$ we find that $\|f\|_{M^*} \geq 1$. We summarize, $f \in M^*$ with $f_{ \mid Z}=0$ and $\|f\|_{M^*}=1$. From Corollary 4.1. there exists $F \in X^*$ with $\|F\|_{X^*}=\|f\|_{M^*}=1$ and $F_{\mid M} = f$ which gives in particular that $F(y)=f(y)=d$ and $F_{ \mid Z}=0$.
\end{proof}
\begin{rem} Yet another consequence of Hahn-Banach Theorem is that if $X$ is a normed space, then $X^*$ seperates the points of $X$, i.e. for every $x_1,x_2 \in X$ with $x_1 \neq x_2$, there exists $f \in X^*$ with $f(x_1) \neq f(x_2)$. In fact, given $x_1,x_2 \in X$ with $x_1 \neq x_2$ we set $y=x_1-x_1 \neq 0$. Then from Corollary 4.2. there exists $f \in X^*$ with $f(y)= \|y\| \neq 0$ which implies that $f(x_1) \neq f(x_2)$. 
\end{rem}
\newpage
\begin{defn} Let $X,Y$ be two normed spaces and $T:X \to Y$ a continuous linear operator. The adjoint operator to $T$, denoted by $T^*$, is a linear map $T^*: Y^* \to X^*$, defined by $T^*(f)=f \circ T$, for all $f  \in Y^*$. 
\end{defn}
\begin{rem} Notice that if $T \in \mathcal{L}(X,Y)$ and $S \in \mathcal{L}(X,Z)$, then we have $(ST)^*= T^*S^*$. 
\end{rem}
\begin{prop} Let $T \in \mathcal{L}(X,Y)$. Then $T^* \in \mathcal{L}(Y^*, X^*)$ with $\|T^*\|= \|T\|$. 
\end{prop}
\subsection{Reflexive normed spaces}
Let $X$ be a normed space. Apart from the dual space $X^*$ also the bidual space $X^{**}$ plays an important role. We construct the canonical inclusion of $X$ in $X^{**}$ as follows: For $x \in X$ we define the linear function $\tilde{x}:X^* \to \mathbb{K}$ through 
\begin{align*}
\tilde{f}(x):=f(x)
\end{align*}
Then indeed $\tilde{x}$ is continuous since $f \in X^*$ is continuous, hence we have $\tilde{x} \in X^{**}$. We define $J_X: X \to X^{**}$ as $J_X(x)= \tilde{x}$ and call it the canonical inclusion of $X$ in $X^{**}$. 
\begin{thm} Let $X$ be a normed vector space over $\mathbb{K}$. Then the canonical inclusion $J_X:X \to X^{**}$ is a linear isometry. 
\end{thm} 
\begin{defn} A normed space $(X, \| \cdot \|)$ is called reflexive, if the map $J_X$ is surjective (and therefore an isometric isomorphism, in particular $X \cong X^{**}$).
\end{defn}
\begin{rem} The space $X^{**}= \mathcal{L}(X^*, \mathbb{K})$ is always a Banach space, hence $X$ reflexive implies that $X$ is also a Banach space, i.e. $X$ is complete. 
\end{rem}
\begin{thm} Let $X$ be a reflexive Banach space and $M \subset X$ a closed linear subspace. Then $M$ is reflexive as well. 
\end{thm}
\begin{thm} Let $X$ be a Banach space. Then $X$ is reflexive if and only if $X^*$ is reflexive. 
\end{thm}
\subsection{Hilbert Space Methods}
Hilbert spaces can be identified with their dual spaces, this is the subject of Riesz's Representation Theorem. 
\begin{thm}[Riesz's Representation Theorem] Let $(H,( \cdot , \cdot))$ be an Hilbert space. Let $v \in H$ be arbitrary. The map $R_H^v : H \to H^*$ defined through $R_H^v(u)=(u,v)=\langle u,v\rangle_{H^*}$ is an anti-linear isometric isomorphism. 
\end{thm}
\begin{rem} The theorem states that every element of $H^*$ can be written uniquely in this form. We also use the notation $(R_H(u))(v)=(u,v)$. 
\end{rem}
\newpage
\begin{cor} Every Hilbert space is reflexive.
\end{cor}
\begin{proof}
We claim that $J_H = R_{H^*} \circ R_H$. Then $J_H$ is surjective, as the composition of two surjective maps. Let $u \in H$ and $f \in H^*$ be arbitrary. Then 
\begin{align*}
((R_{H^*} \circ R_H)(u))(f)&= (R_{H^*}(R_H(u))(f)= \langle R_H(u),f \rangle_{H^*} \\
 & = \langle R_H^{-1}f, u \rangle_H = ( R_H(R_H^{-1} f))(u)=f(u)=(J_Hu)(f)
\end{align*}
\end{proof}
\section{Baire Category Theorem and Consequences}
\subsection{Theorems of Baire and of Banach-Steinhaus}
\begin{lem} Let $(X, \tau)$ be a topological space The following statements are equivalent.
\begin{enumerate}
\item Let $(A_i)_{i \in \mathbb{N}}$ be a sequence of closed sets in $X$. If the interior of each $A_i$ is empty, then also the interior of $\cup_{j=1}^\infty A_i$ is empty. 
\item Let $(B_i)_{i \in \mathbb{N}}$ be a sequence of open sets in $R$. If each $B_i$ is dense in $X$, then also $\cap_{j=1}^\infty B_j$ is dense in $X$. 
\end{enumerate}
\end{lem}
\begin{proof}
The equivalence follows from the remark that a set is dense if and only if its complement has an empty interior, then apply De Morgan. 
\end{proof}
\begin{defn} A topological space $(X, \tau)$ is called a Baire's space, if condition 1) or condition 2) (and thus both) are satisfied. 
\end{defn}
\begin{thm}[Baire] Every complete metric space is a Baire space. 
\end{thm}
\begin{proof}
Let $(X,d)$ be a complete metric space with $X \neq \emptyset$. Let $(B_i)_{i \in \mathbb{N}}$ be a sequence of open dense sets in $X$. We have to show that $L:= \cap_{j=1}^\infty B_j$ is again dense in $X$. To this end, it is enough to show that for every non-empty open set $G \subset X$, we have $G \cap L \neq \emptyset$. 
\\\\
Since $B_1$ is open and dense, $G \cap B_1 \neq \emptyset$ and open. Hence, we can find $\epsilon_1 \in (0, 1 ]$ and $x_1 \in X$ such that 
\begin{align*}
\overline{B_{\epsilon_1}(x_1)} \subset B_1 \cap G.
\end{align*}
Since $B_2$ is open and dense, $B_{ \epsilon_1}(x_1) \cap B_2$ is also non-empty and open. Hence we can find $\epsilon_2 \in (0, 1/2]$ and $x_2 \in X$ with 
\begin{align*}
\overline{B_{\epsilon_2}(x_2)} \subset B_{ \epsilon_1} (x_1) \cap B_2.
\end{align*}
Iteratively, we find $\epsilon_n \in (0, 1/n]$ and $x_n \in X$ with 
\begin{align*}
\overline{B_{ \epsilon_n}(x_n)} \subset B_{ \epsilon_{n-1}}(x_{n-1}) \cap B_n.
\end{align*}
The sequence $(x_n)_{n \in \mathbb{N}}$ is then a Cauchy sequence in $X$. In fact, by construction we have for every $m \geq n$ that $B_{ \epsilon_m}(x_m) \subset B_{ \epsilon_n}(x_n)$ and therefore $d(x_n,x_m) \leq \epsilon_n \leq 1/n \to 0$ as $n,m \to \infty$. Since $X$ is complete, the limit $x= \lim x_n$ exists. 
\\\\
From $x_k \in \overline{B_{ \epsilon_n}(x_n)}$ for all $k \geq n$, we deduce that $x \in \overline{B_{ \epsilon_n}(x_n)}$ for all $n \in \mathbb{N}$. Hence 
\begin{align*}
x \in G \cap \bigcap_{j=1}^\infty B_j = G \cap L \implies G \cap L \neq \emptyset. 
\end{align*}
\end{proof}
\noindent As a first application of Baire's Theorem, we prove the Theorem of Banach-Steinhaus. 
\begin{thm}[Banach-Steinhaus] Let $X$ be a Banach space, $Y$ be a normed space and $\mathcal{F} \subset \mathcal{L}(X,Y)$. Assume that for all $x \in X$, there exists $c_x \geq 0$ with 
\begin{align*}
\sup_{T \in \mathcal{F}} \| Tx\| \leq c_x.
\end{align*}
Then there exists $c \geq 0$ with 
\begin{align*}
\sup_{T \in \mathcal{F}} \| T \|  \leq c. 
\end{align*}
\end{thm}
\begin{proof}
For $k \in \mathbb{N}$ we consider the set 
\begin{align*}
A_k := \lbrace x \in X : \| Tx \| \leq k \text{ for all } T \in \mathcal{F} \rbrace.
\end{align*}
Then $A_k$ is closed for all $k \in \mathbb{N}$. In fact, if $(x_j)_{j \in \mathbb{N}}$ denotes a sequence in $A_k$ with $x_j \to x$ in $X$, then we have for every $T \in \mathcal{F}$,
\begin{align*}
\|Tx\| \overset{\Delta} \leq \| Tx_j\| + \|T(x-x_j)\| \leq k + \|T\| \|x-x_j\|   \\
\implies \|Tx\| \leq k \text{ for $j \in \mathbb{N}$ sufficiently large}
\end{align*}
Since this holds for all $T \in \mathcal{F}$, we conclude that 
\begin{align*}
\sup_{T \in \mathcal{F}} \|Tx \| \leq k, \text{ i.e. } x \in A_k \text{ and thus } A_k \text{ is closed}. 
\end{align*}
We have that 
\begin{align*}
\bigcup_{k=1}^\infty A_k = X.
\end{align*}
From Baire's Category Theorem (see condition 1, contraposition of that) we obtain that there exists $k_0 \in \mathbb{N}$ with $A_{k_0}^\circ \neq emptyset$.
\newpage We find therefore $x_0 \in X$ and $\epsilon_0 >$ with $\overline{B_{ \epsilon_0}(x_0)} \subset A_{k_0}^\circ$. Hence for any $y \in X$ with $\| y \| \leq \epsilon_0$ we get 
\begin{align*}
\|Ty\| \leq \| T(y+x_0)\| + \|Tx_0\| \leq k_0 + c_{x_0}, \text{ for all } T \in \mathcal{F}.
\end{align*}
Consequently we have for all $y \in X$ with $y \neq 0$ and for all $T \in \mathcal{F}$,
\begin{align*}
\|Ty \| = \left\| T \left( \frac{y}{\|y\|} \epsilon_0 \frac{\|y \|}{\epsilon_0} \right) \right\| = \frac{\|y\|}{\epsilon_0} \left\| T \left( \frac{y}{\|y\|} 	\epsilon_0 \right) \right\| \leq \frac{k_0 + c_{x_0}}{\epsilon_0} \|y\|.
\end{align*}
We conclude that
\begin{align*}
\|T\| = \sup_{\|y\| \leq 1} \|Ty\| \leq \frac{k_0+c_{x_0}}{\epsilon_0}=:c
\end{align*}
\end{proof}
\subsection{The open map and the closed graph theorems}
\begin{thm}[Open map theorem] Let $X,Y$ be Banach spaces and let $T \in \mathcal{L}(X,Y)$ be surjective (i.e. $T(X)=Y)$. Then $T(U)$ is open in $Y$ for all open sets $U \subset X$. 
\end{thm}
\noindent An equivalent formulation of the open map theorem is given by the following inverse map theorem.
\begin{thm}[Inverse Map Theorem] Let $X,Y$ be Banach spaces, $T:X \to Y$ a continuous linear bijection. Then $T^{-1} \in \mathcal{L}(X,Y)$ i.e. is also linear and continuous, which implies that there exists $c>0$ with 
\begin{align*}
\frac{1}{c}\|x\| \leq \|Tx\| \leq c\|x\|, \text{ for all } x \in X. 
\end{align*}
\end{thm}
\begin{proof}
Clearly, $T^{-1}$ is well-defined and linear. From the inverse mapping theorem we know that there exists $r>0$ such that $T(B_1^X(0)) \supset B_r^Y(0)$, i.e. $T^{-1}(B_r^Y(0)) \subset B_1^X(0)$. This implies that for arbitrary $y \in Y$ we have 
\begin{align*}
\|T^{-1} y \| = \frac{2\|y\|}{r} \left\| T^{-1} \frac{ry}{2\|y\|} \right\| \leq \frac{2}{r} \|y\|. 
\end{align*}
Hence $T^{-1}$ is bounded and therefore continuous. 
\end{proof}
\noindent A consequence of the inverse map theorem is the following closed graph theorem. 
\begin{thm}[Closed Graph Theorem] Let $X,Y$ be Banach spaces, $T: X \to Y$ linear. Then are the following statements equivalent: 
\begin{enumerate}
\item graph$(T)=\lbrace (x,TX) \in X \times Y : x \in X \rbrace$ is closed in $X \times Y$ w.r.t. the norm $\|(x,y)\|_{X \times Y} := \|x\|_X + \|y\|_Y$.
\item $T$ is continuous.
\end{enumerate}
\end{thm}
\newpage
\begin{proof}
"2) $\implies$ 1)". Let $(x,y) \in \overline{\text{graph}(T)}$. Then there exists a sequence $(x_k)_{k \in \mathbb{N}}$ in $X$ with $(x_k, T_{x_k}) \to (x,y)$. Hence $x_k \to x$ in $X$ and $T_{x_k} \to y$ in $Y$. Since $T$ is continuous, we have that $y=Tx$ and thus $(x,y) \in \text{graph}(T)$. Thus graph$(T)$ is closed. 
\\\\
"1) $\implies$ 2)" Define $\phi : X \to X \times Y$ through $\phi(x)= (x,Tx)$. The image of $\phi$ is then given by 
\begin{align*}
R(\phi):= \lbrace \phi(x) : x \in X \rbrace = \text{graph}(T)
\end{align*}
and therefore, by assumption, is a closed subset of $X \times Y$. Hence $R( \phi)$ is a Banach space w.r.t. the norm $\|(x,y)\|= \|x\|_X + \|y\|_Y$ and the map $\phi: X \to R(\phi)$ is a linear bijection between Banach spaces. We want to show that $\phi$ is continuous, to this end we discuss it's inverse. The inverse map $\phi^{-1} : R( \phi) \to X$ is given by $\phi^{-1}(x,Tx)=x$ is also a linear bijection. Moreover, since
\begin{align*}
\|\phi^{-1}(x,Tx)\|_X = \|x\|_X \leq \|(x,Tx)\|_{X \times Y}
\end{align*}
we have that $\phi^{-1}$ is a continuous linear bijection. The inverse map theorem implies that also $(\phi^{-1})^{-1}=\phi$ is continuous. Since the projection $p: X \times Y \to X$, defined as $p(x,y)=y$ is clearly continuous, we conclude that $T = p \circ \phi$ is continuous as well. 
\end{proof}
\begin{defn} Let $X$ be a Banch space. A continuous map $P: X \to X$ with $P^2=P \circ P = P$ is called a projection on $X$. It is easy to see that if $P$ is a projection, then also $1-P$ is a projection. 
\end{defn}
\begin{thm} Let $X$ be a Banach space and $P \in \mathcal{L}(X)$ a projection. Then $X=P(X) \oplus (1-P)(X)$. 
\end{thm}
\begin{proof}
Let us define $A= \text{Ran } P = \lbrace Px: x \in X\rbrace$ and $B= \text{Ran } (1-P)= \lbrace (1-P)x: x \in X \rbrace$. Then we have:
\\
$\bullet$ $A$ is closed. In fact, if $(y_k)_{k \in \mathbb{N}}$ is a sequence in $A$ with $y_k \to y$ in $X$, then by definition $y_k = P{x_k}$ for appropriate $x_k \in X$. Hence
\begin{align*}
Py_k = P(P{x_k})=P^2x_k = Px_k = y_k \text{ for all } k \in \mathbb{N}.
\end{align*}
if we now let $k \to \infty$ we find, since $P \in \mathcal{L}(X)$ is continuous, that $Py=y$ and thus $y \in A$ i.e. $A$ is closed. 
\\
$\bullet$ $B$ is also closed, using the same argument as above with $P$ replaced by $1-P$. 
\\\\
$\bullet$ $X=A+B$ is clear, since $x=Px + x -Px = Px + (1-P)x$.
\\\\
$\bullet$ $A \cap B = \lbrace 0 \rbrace$. If $x \in A$ then we have $x=Px$ (because if $x=Py$ then $Px=P^2y=Py=x$) and if $x \in B$ we have $x=(1-P)x$, thus $x=Px=P(1-P)x=0$. 
\end{proof}
\newpage
It follows from the last theorem that finite dimensional subspaces always have a topological complement.
\begin{thm} Let $X$ be a Banach space, $A \subset X$ a finite dimensional linear subspace. Then there exists a closed subspace $B \subset X$ with $X = A \oplus B$. 
\end{thm}
\section{Weak Topologies on normed spaces}
\subsection{The weak and the weak-* topologies}
Let $X$ be a set and let $\mathcal{F}$ denote a family of maps $f: X \to Y_f$, where $Y_f$ is a topological space. On $X$, we introduce the topology $\tau_\mathcal{F}$, defined as the smallest topology on $X$ so that all functions in $\mathcal{F}$ are continuous. We know that the arbitrary union of a family of topologies on $X$ is again a topology on $X$, thus for 
\begin{align*}
\mathcal{S}:= \lbrace f^{-1}(V): V \subset Y_f \text{ open}, f \in \mathcal{F} \rbrace
\end{align*}
we define
\begin{align*}
\tau_\mathcal{F} = \bigcap \lbrace \tau : \tau \text{ is a topology on $X$ and $\mathcal{S} \subset \tau$} \rbrace 
\end{align*}
Then $\tau_\mathcal{F}$ is the smallest topology on $X$, with the property that all maps in $\mathcal{F}$ are continuous. As always, we care about the Hausdorff property. 
\begin{prop} Let $X$ be a set and $\mathcal{F}$ a family of maps $f: X \to Y_f,$ with $Y_f$ Hausdorff for every $f \in \mathcal{F}$. Then $\mathcal{F}$ separates the points of $X$ and $(X, \tau_\mathcal{F})$ is again a Hausdorff space. 
\end{prop}
\noindent We now specialize our construction to the case of normed spaces.
\begin{defn} Let $X$ be a normed space, $X^*$ its dual space and $X^{**}$ the bidual space. The smallest topology on $X$, with the property that all $f \in X^*$ are continuous is called the weak topology on $X$, it is denoted by $\tau_W$. \\
Since $X^*$ is again a normed space, we can introduce also on $X^*$ the weak topology, defined as the smallest topology with the property that all $f \in X^{**}$ are continuous. Moreover we define on $X^*$ the weak-* topology, denoted by $T_W^*$, it is defined as the smallest topology such that all $f \in J_X(X) \subset X^{**}$ are continuous where $J_X:X \to X^{**}$ is the canonical inclusion. 
\end{defn}
\begin{rem} \
\begin{enumerate}
\item On $X$ we have $\tau_W \subset T_{ \| \cdot \|}$, i.e. convergence (w.r.t. the norm on $X$) always implies weak convergence. This follow because by definition $f \in X^*$ is always continuous w.r.t. $\tau_{\| \cdot\|}$ and because $T_W$ is the smallest topology on $X$ such that $f \in X^*$ is continuous. 
\item On $X^*$ we have $\tau_W^* \subset \tau_W \subset \tau_{\| \cdot \|}$, because $J_X(X) \subset X^{**}$. 
\item If $X$ is reflexive, then $J_X(X) = X^{**}$ and $\tau_W= \tau_W^*$ as topologies on $X^*$. 
\end{enumerate}
\end{rem}
\newpage
\noindent On infinite dimensional vector spaces, norm- and weak topologies are never the same (i.e. not equivalent), as the next Lemma entails
\begin{lem} Let $X$ be a normed space. Then $\tau_W = \tau_{\| \cdot \|}$ if and only if $\dim X < \infty$. If $\dim X = \infty$, then every $\tau_W$-open neighbourhood of $0$ contains an infinitely dimensional linear subspace of $X$. 
\end{lem}
\subsection{The notion of weak convergence}
Let $X$ be a normed space. As every topology, the weak topology $\tau_W$ induces a notion of convergence of sequences in $X$. In this context, we say that a sequence $x_k$ in $X$ converges weakly towards $x \in X$ and we write $x_k \rightharpoonup x.$, if the sequence converges w.r.t. $\tau_W$. 
\begin{rem} \
\begin{enumerate}
\item Since $(X, \tau_W)$ and $(X^*, \tau_W^*)$ are both Hausdorff, weak and weak-* limits are always unique if they exist. 
\item Since $\tau_W \subset \tau_{\| \cdot \|}, \ x_k \to x$ w.r.t. $\tau_{\| \cdot \|}$ always implies $x_k \wto x$. Analogously, since $\tau_W^* \subset \tau_W \subset \tau_{\| \cdot \|}$ both, norm and weak convergence, imply weak-* convergence on $X^*$. 
\end{enumerate}
\end{rem}
\begin{lem}[Characterisation of weak convergence] Let $X$ be a normed space. Let $(x_k)_{k \in \mathbb{N}}$ be a sequence in $X$, then we have $x_k \wto x$ if and only if $f(x_k) \to f(x)$ for all $f \in X^*$.
\end{lem}
\noindent The Lemma above says that we can identify weak convergence with convergence w.r.t. the norm on $\mathbb{K}$. 
\begin{lem}[Characterisation of weak-* convergence] Let $(f_i)_{i \in \mathbb{N}}$ be a sequence in $X^*$. Then $f_i$ convergence towards $f \in X^*$ w.r.t. to $\tau_W^*$ if and only if $f_i(x) \to f(x)$ for all $x \in X$.
\end{lem}
\noindent Despite the fact, that in general weak convergent sequences do not converge w.r.t. the norm topology, they are always bounded. This is a consequence of the Banach-Steinhaus theorem and subject of the next Proposition. 
\begin{prop} Let $X$ be a normed space, $(x_k)_{k \in \mathbb{N}}$ be a sequence in $X$ with $x_k \wto x$. then $(x_k)_{k \in \mathbb{N}}$ is bounded and moreover we have 
\begin{align*}
\| x \| \leq \liminf_{k \to \infty} \|x_k\|
\end{align*}
\end{prop}
\newpage
\subsection{Weak-* topology and compactness}
For finite dimensional normed spaces, it is very easy to characterize compact sets thanks to the Heine Borel Theorem, a set $M$ is compact if and only if $M$ is bounded and closed. For infinite dimensional normed spaces, this is no longer true. We have also seen in Proposition \ref{compactunitball} that the closed unit ball $\overline{B_1(0)}$ is compact if and only if $\dim X < \infty$, in particular in the infinite dimensional case $\overline{B_1(0)}$ is neither covering nor sequentially compact. 
\\\\
Weak topologies are important because often they allow us to "restore" the compactness of the closed ball. The statement that the closed unit ball in $X^*$ is compact w.r.t. $\tau_W^*$ is known as the Banach-Alaoglu Theorem. 
\begin{thm}[Tychonov Theorem] Let $(X_\lambda)_{ \lambda \in \Lambda}$ be a family of compact sets. Then the product $X= \prod_{\lambda \in \Lambda}  X_\lambda$ is compact, w.r.t. the product topology defined on $X$. 
\end{thm}
\begin{rem} The product topology on $X$ is generated by product sets having the form 
\begin{align*}
\prod_{ \lambda \in \Lambda} U_\lambda = \lbrace (x_\lambda)_{ \lambda \in \Lambda} :  x_\lambda \in U_\lambda \rbrace, \text{ where } U_\lambda \subset X_\lambda \text{ are open for all } \lambda \in \Lambda. 
\end{align*}
Equivalently, the product topology is the smallest topology on $X$, so that all projections $p_\lambda : X \to  X_\lambda$, defined through $p((x_\lambda)_{ \lambda \in \Lambda}) = x_\lambda$, are continuous. 
\end{rem}
\begin{thm}[Banach-Alaoglu] Let $X$ be a normed space and $X^*$ its dual space. Then the closed unit ball $k_{X^*} = \lbrace f \in X^* : \| f \| \leq 1 \rbrace$ in $X^*$ is compact w.r.t the weak-* topology $\tau_W^*$. 
\end{thm}
\noindent Since in general the weak-* topology is not metrizable, the notion of sequentially compactness is not equivalent to the notion of (covering) compactness. In fact, in general the unit ball in $X^*$ is not sequentially compact w.r.t $\tau_W^*$. The next theorem shows, however, that the unit ball in $X^*$ is sequentially compact, under the assumption that $X$ is also seperable. 
\begin{thm} Let $X$ be a separable normed space. Then the closed unit ball $k_{X^*}$ in $X^*$ is sequentially compact, w.r.t. the $\tau_W^*$ topology. 
\end{thm}
\subsection{Reflexivity and compactness}
In this section, we consider reflexive Banach spaces and we show that, under appropriate assumptions, the closed unit ball in $X$ is compact and sequential compact w.r.t the $\tau_W$ topology. 
\begin{prop} Let $X$ be a Banach space. Then $X$ is reflexive if and only if $k_X= \lbrace x \in X : \|x\| \leq 1\rbrace$ is compact w.r.t the $\tau_W$ topology. 
\end{prop}
\newpage
\noindent For reflexive Banach spaces, $k_X$ is not only compact, it is also sequentially compact w.r.t. the topology $\tau_W$. 
\begin{thm} Let $X$ be a reflexive Banach space. Then the closed unit ball $k_X$ is sequentially compact w.r.t. the topology $\tau_W$. In other words, every bounded sequence in $X$ has a $\tau_W$-convergent subsequence. 
\end{thm}
\subsection{Weak topology and convexity}
A weakly convergent sequence $(x_n)_{n \in \mathbb{N}}$ in a Banach space $X$ has one additional neat property, that is often useful. If $x_n$ converges weakly, there exists namely another sequence, consisting of convex combinations of the elements of $x_n$, converging strongly (i.e. w.r.t. the norm topology). This results is known as the Lemma of Mazur. 
\begin{thm} Let $X$ be a normed space over $\mathbb{K}$ and $A \subset X$ covex. Then $\overline{A}^{\tau_{\| \cdot \|}} = \overline{A}^{ \tau_W}$, i.e. the closure of $A$ w.r.t. the (strong) norm topology is the same as the closure of $A$ w.r.t. the weak topology. 
\end{thm}
\begin{cor}[Mazur's Lemma] Let $(x_n)_{n \in \mathbb{N}}$ be a sequence in a Banach space $(X, \| \cdot \|)$, converging weakly towards $x$. Then there exists a sequence $(y_k)_{k \in \mathbb{N}}$ so that every $y_k$ is a finite convex combination of the elements $x_n$, i.e. for every $k \in \mathbb{N}$ we have
\begin{align*}
&y_k = \sum_{n \in \mathbb{N}} t_n^{(k)} x_n, \text{ s.t. for every fixed $k \in \mathbb{N}$, } t_n^{(k)} \neq 0 \text{ for only finitely many n} \\
&\text{and } \sum_{n \in \mathbb{N}} t_n^{(k)} = 1
\end{align*}
and $y_k \to x$ wr.t. the norm topology (i.e. strongly). 
\end{cor}
\begin{proof}
Let $A= \lbrace$finite convex combinations of $(x_n)_{n \in \mathbb{N}} \rbrace$. Then $A$ is convex and thanks to the above theorem we have $\overline{A}^{ \tau_{ \| \cdot \|}} = \overline{A}^{ \tau_W}$. Since $x_n \wto x $ we have $x \in \overline{A}^{ \tau_W}=\overline{A}^{ \tau_{ \| \cdot \|}}$. Hence, there exists a sequence $(y_k)_{k \in \mathbb{N}}$ in $A$ such that $y_k \to x$, because the norm closure of $A$ is exactly the set of all norm limits of sequences in $A$. 
\end{proof}
\end{document}