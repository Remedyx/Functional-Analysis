
\documentclass[12pt,a4paper]{article}
\usepackage[utf8]{inputenc}
\usepackage{amsmath}
\usepackage{amsfonts}
\usepackage{amssymb}
\usepackage{amsthm}
\usepackage{tikz-cd}
\usepackage{graphicx}
\author{Marco Bertenghi}

\newtheorem{lem}{Lemma}[section]
\newtheorem{thm}{Theorem}[section]
\newtheorem{prop}{Proposition}[section]
\newtheorem{cor}{Corollary}[section]
\theoremstyle{definition}
\newtheorem{defn}{Definition}[section]
\newtheorem{exmp}{Example}[section]
\newtheorem{rem}{Remark}[section]

\newcommand{\wto}{\rightharpoonup}

%http://tikzcd.yichuanshen.de/#eyJub2RlcyI6W3sicG9zaXRpb24iOlsxLDBdLCJ2YWx1ZSI6IlgifSx7InBvc2l0aW9uIjpbNCwwXSwidmFsdWUiOiJZIn0seyJwb3NpdGlvbiI6WzQsMl0sInZhbHVlIjoiWV57Kip9In0seyJwb3NpdGlvbiI6WzEsMl0sInZhbHVlIjoiWF57Kip9In0seyJwb3NpdGlvbiI6WzAsMF0sInZhbHVlIjoiWV4qIn0seyJwb3NpdGlvbiI6WzUsMl0sInZhbHVlIjoiWF4qIn0seyJwb3NpdGlvbiI6WzAsMl0sInZhbHVlIjoiWV57KioqfSJ9XSwiZWRnZXMiOlt7ImZyb20iOjAsInRvIjoxLCJ2YWx1ZSI6ImcifSx7ImZyb20iOjEsInRvIjoyLCJsYWJlbFBvc2l0aW9uIjoibGVmdCIsInZhbHVlIjoiSl9ZIiwidGFpbCI6Imhvb2thbHQifSx7ImZyb20iOjMsInRvIjoyLCJsYWJlbFBvc2l0aW9uIjoibGVmdCIsInZhbHVlIjoiZ157Kip9In0seyJmcm9tIjowLCJ0byI6MywidmFsdWUiOiJKX1giLCJsYWJlbFBvc2l0aW9uIjoicmlnaHQiLCJ0YWlsIjoiaG9va2FsdCIsImhlYWQiOiJ0d29oZWFkcyJ9LHsiZnJvbSI6MCwidG8iOjQsInZhbHVlIjoiZiIsImxhYmVsUG9zaXRpb24iOiJyaWdodCIsInRhaWwiOiJob29rIn0seyJmcm9tIjoyLCJ0byI6NSwibGFiZWxQb3NpdGlvbiI6ImxlZnQiLCJ2YWx1ZSI6ImZeKiIsInRhaWwiOiJob29rYWx0In0seyJmcm9tIjozLCJ0byI6NiwiaGVhZCI6bnVsbCwibGFiZWxQb3NpdGlvbiI6InJpZ2h0IiwidmFsdWUiOiJmXnsqKn0ifSx7ImZyb20iOjQsInRvIjo2LCJ0YWlsIjoiaG9va2FsdCIsInZhbHVlIjoiSl97WV4qfSIsImxhYmVsUG9zaXRpb24iOiJyaWdodCJ9XX0=

\begin{document}
\section{Weak convergence without norm convergence}
Let $f \in C_c^\infty( \mathbb{R})$, $f \neq 0$ and $1 \leq p < \infty$. We define the following sequences $g_n(x):= f(x-n), h_n(x):= n^{-1/p}f(x/n)$ and $k_n(x):= f(x) e^{inx}$. for $n \in \mathbb{N}$ and $x \in \mathbb{R}$.
\\\\
Show that for $1 <p < \infty, g_n, h_n$ and $k_n$ converge weakly in $L^p( \mathbb{R})$ to zero, but they do not converge with respect to the norm. Investigate what happens for $p=1$ as well.
\begin{proof}
Recall the characterization of weak convergence:
\begin{lem}[Characterization of weak convergence] Let $X$ be a normed space, $x \in X$ and $(x_n)_{n \in \mathbb{N}}$ a sequence in $X$. Then we have $x_n \wto x$ in $X$ if and only if $f(x_n) \to f(x)$ for all $f \in X^*$. 
\end{lem}
\noindent From the Lemma we immediately recover the fact, that strong convergence (i.e. convergence with respect to the norm $\|x_n-x\|_X \to 0$ as $n \to \infty$) implies weak convergence. Moreover we recall that $X^* = \mathcal{L}(X, \mathbb{K})$ are continuous linear functionals from $X$ to $\mathbb{K}$. 
 \\
\\
In our case $X= L^p( \mathbb{R})$ (for some $p \in (1, \infty)$) is indeed a normed space, moreover we have $C_c( \mathbb{R}) \subset L^p( \mathbb{R})$ for all $1 \leq p < \infty$ (in fact it is a dense subset), thus our sequences are also well defined in $L^p( \mathbb{R})$. It is our goal to show that although the sequences do not convergence with respect to the norm on $L^p(\mathbb{R})$ they do nevertheless converge weakly (towards 0). 
\\\\
Furthermore we must recall another important result which discusses the duality of the $L^p$ spaces:
\begin{thm} Let $1 \leq p < \infty, \ 1 < q \leq \infty$ with $1/p + 1/q =1 $ and let $( \Omega, \mathcal{A}, \mu)$ be a measure space. If $p=1$, we assume additionally the measure space $(\Omega, \mathcal{A}, \mu)$ to be $\sigma$-finite. Then the map 
\begin{align*}
\phi : \begin{cases} L^q( \Omega, \mathcal{A}, \mu) & \longrightarrow (L^p( \Omega, \mathcal{A}, \mu))^* \\ f & \longmapsto \phi_f: \begin{cases}L^p & \longrightarrow \mathbb{K} \\ g & \longmapsto \phi_f(g):= \displaystyle \int_\Omega \overline{f(x)} g(x) d \mu(x)  \end{cases} \end{cases}
\end{align*}
is an (anti-linear) isometric isomorphism, in other words $L^p( \Omega, \mathcal{A}, \mu)^* \cong L^q( \Omega, \mathcal{A}, \mu)$ 
\end{thm}
The Theorem tells us how we can identify the elements of $(L^p)^*$. More explicitly, every continuous linear functional from $L^p(\Omega, \mathcal{A}, \mu)$ to $\mathbb{K}$ for $1 \leq p < \infty$ can be written as 
\begin{align*}
L(f) = \int_\Omega g(x)f(x) d \mu(x)
\end{align*}
for a $g \in L^q( \Omega,  \mathcal{A}, \mu)$ where $p,q$ are conjugate. 
\\\\
Let us now fix $p \in (1, \infty)$ and $q$ conjugate to $p$. Then we know that all $\phi \in (L^p)^*$ can be written in the form above for some $g \in L^q$. We thus obtain:
\begin{align*}
\phi(g_n) = \int_\mathbb{R} g(x) g_n(x)dx = \int_\mathbb{R}g(x) f(x-n) dx 
\end{align*}
We know that $f$ is compactly supported, in particular we have supp($f) \subset [-a,a]$ for some positive $a \in \mathbb{R}$. In particular we have that the support of $f(\cdot -n)$ is contained in $[-a +n, a+n]$. Using this observation and Hölder's inequality we obtain that
\begin{align*}
|\phi (g_n)| &\leq \int_\mathbb{R}|g(x)f(x-n)|dx = \int_\mathbb{R} g(x) f(x-n)\mathcal{X}_{\text{supp}(f)} dx \\
&= \int_\mathbb{R} |g(x)f(x-n)|\mathcal{X}_{[-a+n,a+n]} dx \\
& \leq \| g \mathcal{X}_{[-a+n, a +n]} \|_q \|f(x-n)\|_p = \|g \mathcal{X}_{[-a+n, a+n]} \|_q \|f\|_p
\end{align*}
For the first term at the end of the inequality we notice that $g(x)^q\mathcal{X}_{[-a+n, a+n]}$ is of course dominated by $g(x)^q$ which is by assumption integrable, further we have as $n$ tends to infinity that $g(x)^q\mathcal{X}_{[-a+n, a+n]} \to 0$ because the indicator function will vanish. Bringing these results together we have
\begin{align*}
|\phi(g_n)| \leq \|g \mathcal{X}_{[-a+n, a+n]} \|_q \|f\|_p \to 0 \text{ as } n \to \infty
\end{align*}
Which translates to $\lim_{n \to \infty} \phi(g_n) = 0 = \phi(0)$ and by the Lemma we obtain that $g_n \wto 0$. 
\\\\
Considering now the sequence $(h_n)_{n \in \mathbb{N}}$, then we observe that the support of $h_n$ is contained in $[-a/n, a/n]$ for some positive $a \in \mathbb{R}$. In the exact same fashion as above (but giving much less detail now) we obtain that 
\begin{align*}
|\phi(h_n)| \leq \|g \mathcal{X}_{[-a/n,a/n]}\|_q \|f\|_p \to 0 \text{ as } n \to \infty
\end{align*}
Because the indicator function will only contain of one point, namely, $\lbrace 0 \rbrace$ and the Lebesgue integral vanishes. This entails that $h_n \wto 0$. 
\\\\
Finally, we obtain in the same fashion
\begin{align*}
\phi(k_n)= \int_\mathbb{R} g(x) f(x)e^{inx} dx
\end{align*}
We recall the Riemann Lebesgue-Lemma which states that for $f \in L^1(\mathbb{R})$ and $\hat{f}$ its Fourier Transform we have that $\hat{f}(z) \to 0$ as $|z| \to \infty$. In our case we have by Hölder's inequality that $f \cdot g \in L^1(\mathbb{R})$ and its Fourier Transform agrees with our expression above, i.e. we have \begin{align*}
\widehat{f\cdot g}(-n) = \phi(k_n) \to 0 \text{ as } |-n| = n \to \infty
\end{align*}
which yields that $k_n \wto 0$. 
\\\\
We now turn out attention towards the convergence in the $L^p$-norm. Let us first consider the case of $p=1$, we then have
\begin{align*}
\|g_n -f \| = \int_{\mathbb{R}} | g_n(x)-f(x)|dx  =  \int_\mathbb{R} | f(x-n)-f(x)| dx  \\
= \int_\mathbb{R} |f(x)-f(x) | dx = 0 
\end{align*}
That is $g_n \to f$ in $L^1(\mathbb{R})$ as $n \to \infty$. 
\\\\
Moreover we have
\begin{align*}
\|h_n-f\|_1 &= \int_\mathbb{R} |h_n(x)-f(x)|dx = \int_\mathbb{R} |n^{-1} f(x/n)-f(x)|dx \\ &= \int_\mathbb{R} |f(x)-f(x)|dx = 0 
\end{align*}
thus, $h_n \to f$ in $L^1( \mathbb{R})$ as $n \to \infty$.
\\\\
Let us now consider the case for $p \in (1, \infty)$. 
\\\\
We already know that strong convergence (i.e. convergence with respect to the norm) implies weak convergence and that in either cases the limits, if they exists, are unique and agree with one another. Since we have already established that the sequences convergence weakly towards $0$, the only possibility for said sequences to converge with respect to the $L^p$-norm is if they would converge to $0$. 
\newpage
However, we easily see that 
\begin{align*}
\|g_n\|_p^p &= \int_\mathbb{R}|f(x-n)|^pdx = \int_\mathbb{R} |f(x)|^p dx = \|f\|_p^p \\
\|h_n\|_p^p & = \int_\mathbb{R}n^{-1} |f(x/n)|^pdx = \int_\mathbb{R} |f(x)|^p dx = \|f\|_p^p \\
\|k_n\|_p^p & = \int_\mathbb{R}\underbrace{|e^{inpx}|}_{=1}|f(x)|^p = \int_\mathbb{R} |f(x)|^p dx = \|f\|_p^p
\end{align*}
Where we used the obvious substitutions in the first two lines (proof Exercise :-D). \\\\
By assumption we have $f \neq 0$ thus none of the above sequences converges in norm to $0$, thus they don't converge in norm at all. 
\end{proof} 
\newpage
\section{Criteria for weak and norm convergence}
\textbf{a)} Let $1 < p < \infty$, $\mathbb{K}$ be $\mathbb{C}$ or $\mathbb{R}$, and let $(x^{(n)})_{n \in \mathbb{N}}$ be a sequence in $\ell^p( \mathbb{K})$. Let $x \in \ell^p(\mathbb{K})$. Show that $(x^{(n)})_{n \in \mathbb{N}}$ converges weakly to $x$ if and only if $(x^{(n)})_{n \in \mathbb{N}}$ is bounded and $x_i^{(n)} \to x_i$ as $n \to \infty$ for all $i \in \mathbb{N}$ 
\begin{proof}
Let us assume that $(x^{(n)})_{n \in \mathbb{N}}$ converges weakly to $x$. Then we know already that the sequence must be bounded by \textbf{Proposition 6.2.2.}
\\
\\
Moreover we know that weak converges holds if and only if $f(x^{(n)}) \to f(x)$ as $n \to \infty$ for all $f \in \ell^p( \mathbb{K})^*$. In particular it must hold true for the projections $p_i : \ell^p( \mathbb{K}) \to \mathbb{K}$ defined by $x= (x^{(n)})_{n \in \mathbb{N}} \mapsto x_i^{(n)}$, for which we already know that they are linear and continuous. 
\\
\\
Thus it follows easily that $x_i^{(n)} \to x_i$ as $n \to \infty$ for all $i \in \mathbb{N}$. 
\\\\
Conversely suppose that $(x^{(n)})_{n \in \mathbb{N}}$ is a bounded sequence and the coordinates $x_i^{(n)}$ converges to $x_i$ as $n \to \infty$ for all $i \in \mathbb{N}$. 
\\\\
Recall from the previous Exercise Sheet 8 Exercise 5, where we have shown the following criteria for weak convergence:
\\\\
\textbf{Sheet 8 Exercise 5:} Let $X$ be a normed space and $Y \subset X^*$ a dense subset. Let $(x_n)_{n \in \mathbb{N}}$ be a sequence in $X$. Then $x_n$ converges weakly to $x \in X$ if and only if $\sup_{n \in \mathbb{N}} \|x_n \| < \infty$ and $f(x_n) \to f(x)$ as $n \to \infty$ for all $f \in Y^*$. 
\\\\
We will apply this exercise to the set $S:= \lbrace f_i \in (\ell^p)^*: i \in \mathbb{N} \rbrace$ where $p \in (1, \infty)$ and $f_i : \ell^p \to \mathbb{K}$ are the canonical projections $t \mapsto t_i$. Indeed we have that $S \subset (\ell^p)^*$, moreover we consider its span $Y:=\text{span}(S) \subset (\ell^p)^*$ and we claim that $Y$ is dense in $(\ell^p)^*$.
\\\\
Since $1 < p < \infty$ we have that $(\ell^p)^*\cong \ell^q$ for some $q$ that satisfies the relation $p^{-1} + q^{-1}=1$. It is therefore enough to show that $S$ is dense in said $\ell^q$. To this extend let $t \in \ell^q$ be arbitrary, by definition we can find for $\epsilon >0$ always an $N \in \mathbb{N}$ such that
\begin{align*}
\sum_{j=N+1}^\infty |t_j|^q < \epsilon^q 
\end{align*}
Let us now choose $y \in Y$ such that $y_j = t_j$ for all $j=1, \dots , N$ and $y_j= 0$ otherwise, then trivially we have $\|t-y\|_q < \epsilon$ which shows the density. 
\end{proof}
\newpage
\textbf{b)} Let $H$ be a Hilbert space, let $x \in H$ and let $(x_n)_{n \in \mathbb{N}}$ be a sequence in $H$. Show that if $x_n$ converges weakly to $x$ and $\lim_{n \to \infty} \|x_n\| = \|x \|$, then $x_n$ converges to $x$ in norm. 
\begin{proof}
Let $(x_n)_{n \in \mathbb{N}}$ be a sequence in the Hilbert space $H$ that converges weakly to some $x$ in $H$. We know that this is the case if and only if $f(x_n) \to f(x)$ as $n \to \infty$ for all $f \in H^*$. 
\\\\
Since this convergence must be satisfied for all $f \in H^*$ it must in particular be the case for \begin{align*}
f_x:= \begin{cases}H & \longrightarrow \mathbb{K} \\
y & \longmapsto \langle x, y \rangle  \end{cases}
\end{align*}
Where we fixed $x \in H$ for the weakly convergent sequence $x_n \wto x$. Indeed $f_x$ is by definition linear (the inner product is linear in its second argument and anti-linear in its first argument). Moreover $f_x$ is a bounded linear operator from $H$ to $\mathbb{K}$ by the Cauchy-Schwarz inequality. 
\\\\
Henceforth $f_x$ is a continuous linear function from $H$ to $\mathbb{K}$, i.e.$ f_x \in H^*$. Consider now:
\begin{align*}
\|x_n-x\|^2 = \langle x_n-x , x_n -x \rangle &= \| x_n \|^2 +\| x\|^2 - \langle x , x_n \rangle  -\langle x_n, x \rangle \\
& = \|x_n\|^2 + \|x\|^2 - \langle x, x_n \rangle - \overline{\langle x, x_n \rangle}  \\
& = \|x_n\|^2 + \|x\|^2 - f_x(x_n) - \overline{f_x(x_n)}
\end{align*}
Since $x_n \wto x$ we have by our efforts above that $\langle x, x_n \rangle = f_x(x_n) \to f(x) = \langle x, x \rangle  = \| x\|^2$ and by assumption we have $\|x_n\| \to \|x\|$ as $n \to \infty$. Thus taking the limit in our derived expression above we conclude that
\begin{align*}
\|x_n- x \|^2 \to 0 \text{ as } n \to \infty 
\end{align*}
That is $x_n$ converges to $x$ in norm. 
\end{proof}
\newpage
\section{Norm for weak-* convergence}
Let $(X, \| \cdot \|)$ be a separable normed vector space. We will also denote by $\| \cdot \|$ the norm on $X^*$. Let $\sigma = (x_n)_{n \in \mathbb{N}}$ be a sequence in the unit ball $S_X:= \lbrace x \in X : \| x \| =1 \rbrace$ such that $\overline{\text{span}(\sigma)}=X$. For $x^* \in X^*$ we define
\begin{align*}
\|x^*\|_\sigma := \sum_{k=1}^\infty 2^{-k} |x^*(x_k)|
\end{align*}
\textbf{a)} Show that $\| \cdot \|_\sigma$ is a norm on $X^*$ and that it satisfies the inequality $\|x^*\|_\sigma \leq \|x^*\|$. 
\begin{proof}
By the fact that $\sigma=(x_n)_{n \in \mathbb{N}}$ is a sequence in the unit ball and the geometric series we obtain that
\begin{align*}
\|x^*\|_\sigma = \sum_{k=1}^\infty 2^{-k} |x^*(x_k)| \leq \sum_{k=1}^\infty 2^{-k} \|x^*\| \|x_k\| = \|x^*\|\sum_{k=1}^\infty 2^{-k} = \|x^*\| 
\end{align*}
Which shows that $\| \cdot \|_\sigma$ is well defined and already gives the claimed inequality. Next we establish that $\| \cdot \|_\sigma$ is indeed a norm.
\\\\
Evidently we have $\| \lambda x^* \|_{ \sigma} = | \lambda | \|x^* \|_\sigma$ for all $\lambda \in \mathbb{K}$ and $x^* \in X^*$. Moreover the triangle inequality follows by the triangle inequality of the norm $| \cdot |$ on $\mathbb{K}$. \\
\\
Finally, we claim that $\|x^*\| = 0 \iff x^* =0$. Indeed the necessary condition is as always trivial, for the sufficient condition we consider 
\begin{align*}
\|x^*\|_\sigma = \sum_{k=1}^\infty 2^{-k} |x^*(x_k)| = 0 \implies |x^*(x_k)|=0 \text{ for all }k \in \mathbb{N}
\end{align*}
Thus we have $x^*(x_k)=0$ for all $k \in \mathbb{N}$. Further, we are given that the span of $\sigma$ is dense in $X$. Thus for an arbitrary $y \in X$ we can always approximate said $y$ with a sequence $(y_n)_{n \in \mathbb{N}}$ in the span of $\sigma$ such that $y_n \to y$ as $n \to \infty$ in the norm of $X$. 
\\\\
Moreover we have by the definition of the span that
\begin{align*}
y_n = \sum_{k=1}^K \lambda_k^{(n)} x_k \text{ for all } n \in \mathbb{N} \text{ and some finite } K \in \mathbb{N}
\end{align*}
where $\lambda_k^{(n)} \in \mathbb{K}$ and $x_k \in X$. By the continuity and linearity of  $x^* \in X^*$ we get that
\begin{align*}
x^*(y) = \lim_{n \to \infty} x^*(y_n) = \lim_{n \to \infty} \sum_{k=1}^K \lambda_k^{(n)} \underbrace{x^*(x_k)}_{=0} =0
\end{align*}
Since $y \in X$ was chosen to be arbitrary we conclude that $x^*=0$ as claimed, thus $\| \cdot \|_\sigma$ is indeed a norm on $X^*$. 
\end{proof}
\newpage
\section{Sequential compactness}
For $\epsilon >0$ and $f \in L^\infty((0,1))$ let 
\begin{align*}
I_\epsilon (f):= \epsilon^{-1} \int_0^\epsilon f(x)dx
\end{align*}
\textbf{a)} Show that $I_\epsilon \in L^\infty((0,1))^*$ with $\|I_\epsilon \|=1$ for every $\epsilon >0$.
\begin{proof}
By the linearity of the integral it is clear that $I_\epsilon$ is a linear functional from $L^\infty(0,1)$ to $\mathbb{K}$. Assume now that $\epsilon \in (0,1)$ is fixed, then we have for all $f \in L^\infty (0,1)$ that
\begin{align*}
\|I_\epsilon (f) \| = \left| \frac{1}{\epsilon} \int_0^\epsilon f(x) dx \right|  \leq \frac{1}{\epsilon} \int_0^\epsilon |f(x)| dx  \leq \frac{1}{\epsilon} \int_0^\epsilon \|f\|_{L^\infty(0,1)} dx \\
= \|f \|_{L^\infty (0,1)}
\end{align*}
Thus $I_\epsilon$ is bounded and therefore as an linear operator also continuous. 
\\\\
Moreover we have
\begin{align*}
\|I_\epsilon\|= \sup_{ \|f \|_\infty = 1} \| I_\epsilon (f) \| = \sup_{ \|f \|_\infty = 1} \left| \frac{1}{\epsilon} \int_0^\epsilon f(x) dx \right| \leq \sup_{ \| f \|_\infty = 1} \frac{1}{\epsilon} \int_0^\epsilon |f(x)| dx  \\
\leq \sup_{ \|f\|_\infty = 1} \frac{1}{\epsilon} \int_0^\epsilon \|f\|_\infty dx  = 1 
\end{align*}
Thus we have $\|I_\epsilon \| \leq 1$. 
\\\\
To establish the other inequality we fix $f\equiv 1 \in L^\infty((0,1))$ and notice that \begin{align*}
I_\epsilon(f) = \frac{1}{\epsilon} \int_0^\epsilon 1 dx = 1
\end{align*}
and trivially $\| f \|_\infty = 1$. We obtain that
\begin{align*}
\|I_\epsilon\| = \sup_{ \|g\|_\infty = 1} |I_\epsilon (g)| \geq |I_\epsilon (f) | = 1
\end{align*}
Thus we conclude that $\|I_\epsilon \| = 1$ 
\end{proof}
\end{document}